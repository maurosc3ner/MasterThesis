% Appendix A

\chapter{frontAlgorithms: A Semi-automatic Coronary Centerline Extraction Framework} \label{AppendixA}

\begin{figure}[ht]
	\centering
		\includegraphics[width=1.3\textwidth]{./Figures/extrac6.png}
	\label{fig:apa_cent1}
\end{figure}
\clearpage

\section{Copyright}
Esteban Correa, Universidad de los Andes or CREATIS-UMR do not offer any support for this product whatsoever. The program is offered free of charge. The executable program is copyrighted freeware by CREATIS-UMR.

\section{Contributors}

\textbf{Esteban Correa}
\href{mailto:esteban.correa@creatis.insa-lyon.fr }{esteban.correa@creatis.insa-lyon.fr }
Master Student for centerline extraction methods and development.
Universidad de los Andes, Bogot\'a D.C. Colombia.
Université Lyon 1, France.

\textbf{Leonardo Fl\'orez}
\href{mailto:florez-l@javeriana.edu.co }{florez-l@javeriana.edu.co }
Associate Professor and main software developer and centerline extraction methods.
TAKINA
Pontificia Universidad Javeriana, Bogot\'a D.C. Colombia.

\textbf{M. Orkisz}
\href{mailto:maciej.orkisz@creatis.insa-lyon.fr }{maciej.orkisz@creatis.insa-lyon.fr }
Associate Professor
CREATIS, CNRS UMR 5220, INSERM U1044.
Université Lyon 1, INSA Lyon. France.

\section{Introduction}
This software is a computer program whose purpose is to evaluate the performance of centerline extraction methods in the context of image processing (and more particularly on CTA images).
The software has been designed for two main purposes:

This method finds a coronary vessel centerline interactively. Two seed points are manually placed by the radiologist and the path is automatically extracted using a minimum cost path approach (Dijkstra’s algorithm). The cost to travel through a voxel is based on a generalized logistic function applied to the non-linear flux vesselness response \citep{Lesage2009a}  of the CTA image. Selective vessel radiuses can be provided to get rid off erroneously connected structures such coronary veins. Results shown robust and accurate centerline extraction in multi-vendor datasets.

Finally, the software allows you to compare the performance of vesselness in a shortest path approach. By default, two vesselness methods are provided \citep{Tek2008,Lesage2009a} with their respective cost functions. A graphical interface enables the algorithm progression.

The software is divided into 2 parts. We can identify:
\begin{itemize}
\item Seeds placement GUI.
\item Centerline Extraction Algorithm.
\end{itemize}

\subsection{System Requirements}

\begin{itemize}
\item At least 2Gb of ram (due to the size of CTA images).
\item Windows or Linux operating systems.
\item CreaTools Framework.
\end{itemize}

\subsection{CreaTools Package Requirements}

\begin{itemize}
\item itkRGC\/frontAlgorithms
\end{itemize}

\section{Point placement}

\textbf{File(s)}
\begin{itemize}
\item fa\_example\_extract\_path\_with\_dijkstra\_06.bbg
\end{itemize}

\textbf{Inputs}
\begin{itemize}
\item Image file (\* .mhd, \* .raw).
\end{itemize}

\textbf{Description}

\begin{figure}[ht]
	\centering
		\includegraphics[width=0.7\textwidth]{./Figures/apa_bbe.png}
	\caption[Radiologist's GUI]{(\textit{Top}) Start point (at ostium). (\textit{Bottom}) end point (distal location) (source: CreaTools screenshot).}
	\label{fig:apa_bbe}
\end{figure}

In order to integrate results from frontAlgorithms extraction with creaCoro aplication. We have build a new bbtk diagram to simulate in some way, how the radiologist works on the images (See Fig. \ref{fig:apa_bbe}). Basically, the user (radiologist) select start and end points of the coronary artery in CTA image (See Fig. \ref{fig:apa_sp1}). 

\begin{figure}[ht]
	\centering
		\includegraphics[width=0.8\textwidth]{./Figures/ShowNPoints1.png}
		\includegraphics[width=0.8\textwidth]{./Figures/ShowNPoints1.png}
	\caption[Radiologist's GUI]{(\textit{Top}) Start point (at ostium). (\textit{Bottom}) end point (distal location) (source: CreaTools screenshot).}
	\label{fig:apa_sp1}
\end{figure}

\section{Centerline Extraction}

\textbf{File(s)}
\begin{itemize}
\item fa\_example\_extract\_path\_with\_dijkstra\_06.bbg
\end{itemize}

\textbf{Inputs}
\begin{itemize}
\item Image file (\* .mhd, \* .raw).
\item Vesselness (offline computed) image file (\* .mhd, \* .raw) .
\item Seed point files (\* .txt)
\end{itemize}

\textbf{Outputs}
\begin{itemize}
\item vtk file containing the centerline parametric path.
\end{itemize}

\textbf{Description}

Currently, you can compute two types of vesselness images (Gulsun, MFlux), command line applications(package\/appli\/examples) are provided for this purpose. Once you have your vesselness image, you can set parameters of the cost function (See section \ref{cent:cost} in order to improve the extraction. Three pre-programmed functions are available. Finally you can launch the extraction method and see the progression interatively (See Fig. \ref{fig:apa_cent1}).

\begin{figure}[ht]
	\centering
		\includegraphics[width=0.5\textwidth]{./Figures/extrac1.png}
		\includegraphics[width=0.5\textwidth]{./Figures/extrac2.png}
		\includegraphics[width=0.5\textwidth]{./Figures/extrac3.png}
		\includegraphics[width=0.5\textwidth]{./Figures/extrac4.png}
	\caption[Centerline Extraction GUI 1]{Extraction progression (source: CreaTools screenshot).}
	\label{fig:apa_cent1}
\end{figure}

\begin{figure}[ht]
	\centering
		\includegraphics[width=0.6\textwidth]{./Figures/extrac6.png}
		\includegraphics[width=0.6\textwidth]{./Figures/extrac7.png}
		\includegraphics[width=0.6\textwidth]{./Figures/extrac8.png}
		\includegraphics[width=0.6\textwidth]{./Figures/extrac9.png}
	\caption[Centerline Extraction GUI 2]{Centerline extracted (source: CreaTools screenshot).}
	\label{fig:apa_cent1}
\end{figure}

