\chapter{Conclusions}\label{conc:conc}

In this thesis we proposed and evaluated an algorithm to detect calcifications in CTA images. In addition, an interactive framework has been developed allowing the evaluation of different vesselness images with a minimum cost algorithm. Also, visualization tools are provided to highlight lesions to support radiologist's diagnosis. Both tools are available in the CreaTools environment\footnote{Medical image processing and visualization software and development tools, CreaTools \href{http://www.creatis.insa-lyon.fr/site/en/softwares-releases}{http://www.creatis.insa-lyon.fr/site/en/softwares-releases}}. In this chapter we summarize the main contributions of the work presented and discuss future research directions.

In Chapter 3 we proposed a flexible cost function to extract centerlines using vesselness flux images. Compared to other centerline extraction methods at CAT08 challenge, more optimization parameters are required to achieve outstanding performance. Moreover, better overlap and vessel accuraccy has been acomplished with this new framework compared to the old team solution. From the vesselness implementation side, further implementation improvements need to be done in order to be a practical solution in clinical procedures.

Chapter 4 presents the lesion detection algorithm. The method is developed using an adapted machine learning method to deal with the skewed dataset. Since not significant difference was founded between RUSBoost and AdaBoost. Supervised methods could be limitated in a CTA assesment approach. Nevertheless, Avoiding false positive bifurcation have been achieved, since we are using a vessel hypothesis and lesions occur next to bifurcations.

The next step towards automation would be the ostium detection for the centerline extraction algorithm, which is able to find the centerlines of all major coronary arteries from a standard clinical CTA dataset. 
Good results for automatic lesion detection extraction have been demonstrated with the results presented in Chapter 6. However, calcified detection of coronary lesions is still not a solved problem. Further research needs to be done in terms of the evaluation framework definition (False negative, etc).