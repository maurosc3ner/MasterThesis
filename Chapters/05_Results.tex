\chapter{Results}
%

\section{Centerline Extraction Experiments}

Both quantitative and qualitative evaluations were conducted to assess the performance of the method. The quantitative evaluation provides an objective evaluation of the method and enables comparison with previously published methods. However, comparison with results in literature should always be done with care, as results have been obtained on different data sets. A limitation of quantitative evaluation is that it requires a set of manually annotated structures, which is time consuming and not feasible for large numbers of 3D data. Therefore, we also conducted qualitative evaluations, as they can be performed in less time and hence on a larger number of data. The quantitative and qualitative evaluation measures used in this work are introduced in Section 2.5.4.


\subsection{Experiment I: MFlux in CAT08 Challenge}

Tables \ref{tb:tb_4_1},\ref{tb:tb_4_2} and \ref{tb:tb_4_3}  presents the results of the semi-automatic coronary centerline extraction on the CTA datasets submitted to the Web-based platform. The following average results were obtained. A 84.3\% overlap with expert human manual annotations was achieved, until the first failure (OF) 65.3\%, in clinically relevant segments (radius $>$ 1.5 mm, OT) 84.4\%. In terms of accuraccy within the vessel (AI) an average of 0.41 mm where obtained. Generally, the method gets a 9$^{th}$ position in the overlaping rank and 12$^{th}$ position for the accuraccy rank in comparison with the CAT08 methods. Visual inspection revealed that most extraction errors occurred at the start of the vessel and near to the ostium. 

\begin{table*}
\scriptsize
\caption{MFlux average overlap per dataset}
\centering
\begin{tabular}{|c|cc|cc|cc|}
\hline
\multicolumn{1}{|c|}{\textbf{Dataset}} &\multicolumn{2}{c|}{\textbf{OV}} &\multicolumn{2}{c|}{\textbf{OF}} &\multicolumn{2}{c|}{\textbf{OT}}\\
\multicolumn{1}{|c|}{\textbf{nr.}} &\multicolumn{1}{c|}{\textbf{\%}} &\multicolumn{1}{c|}{\textbf{score}} &\multicolumn{1}{c|}{\textbf{\%}} &\multicolumn{1}{c|}{\textbf{score}} &\multicolumn{1}{c|}{\textbf{\%}} &\multicolumn{1}{c|}{\textbf{score}}\\
\hline
0&96.2&73.7&69.5&65.2&96.2&73.7\\
1&95.0&47.6&95.0&47.5&95.0&47.5\\
2&99.4&74.8&92.2&71.4&99.4&74.8\\
3&64.5&49.9&47.2&43.5&64.5&49.6\\
4&70.3&58.5&88.8&67.7&71.0&60.7\\
5&95.8&56.8&40.4&20.2&95.8&48.1\\
6&76.9&63.9&73.6&64.1&77.1&63.8\\
7&78.4&40.4&16.7&9.0&78.4&39.7\\
\hline
\textbf{Avg.}&\textbf{84.3}&\textbf{60.4}&\textbf{65.3}&\textbf{50.7}&\textbf{84.4}&\textbf{59.2}\\
\hline
\end{tabular}
\vspace{-0.3cm}
\label{tb:tb_4_1}
\normalsize
\end{table*}

\begin{table*}
\scriptsize
\caption{MFlux average accuracy per dataset}
\centering
\begin{tabular}{|c|cc|}
\hline
\multicolumn{1}{|c|}{\textbf{Dataset}} &\multicolumn{2}{c|}{\textbf{AI}}\\
\multicolumn{1}{|c|}{\textbf{nr.}} &\multicolumn{1}{c|}{\textbf{mm}} &\multicolumn{1}{c|}{\textbf{score}}\\
\hline
0&0.43&33.8\\
1&0.42&30.6\\
2&0.38&27.6\\
3&0.49&29.6\\
4&0.35&27.3\\
5&0.43&34.5\\
6&0.37&27.2\\
7&0.43&24.3\\
\hline
\textbf{Avg.}&\textbf{0.41}&\textbf{29.2}\\
\hline
\end{tabular}
\vspace{-0.3cm}
\label{tb:tb_4_2}
\normalsize
\end{table*}

\begin{table*}
\scriptsize
\caption{MFlux Summary}
\centering
\begin{tabular}{|c|ccc|ccc|}
\hline
\multicolumn{1}{|c|}{\textbf{Measure}} &\multicolumn{3}{c|}{\textbf{\% / mm}} &\multicolumn{3}{c|}{\textbf{score}}  \\
\multicolumn{1}{|c|}{\textbf{}} &\multicolumn{1}{c|}{\textbf{min.}} &\multicolumn{1}{c|}{\textbf{max.}} &\multicolumn{1}{c|}{\textbf{avg.}} &\multicolumn{1}{c|}{\textbf{min.}} &\multicolumn{1}{c|}{\textbf{max.}} &\multicolumn{1}{c|}{\textbf{avg.}}\\
\hline
OV& 8.5\%&100.0\%&84.3\%& 5.8&100.0&60.4\\
OF&10.1\%&100.0\%&65.3\%& 5.0&100.0&50.7\\
OT& 8.5\%&100.0\%&84.4\%& 5.4&100.0&59.2\\
AI&0.28 mm&0.56 mm&0.41 mm&21.7&41.3&29.2\\
\hline
\end{tabular}
\vspace{-0.3cm}
\label{tb:tb_4_3}
\normalsize
\end{table*}

\subsection{Experiment II: MFlux VS Gulsun \& Tek itkRGC}

Tables \ref{tb:tb_4_4},\ref{tb:tb_4_5} and \ref{tb:tb_4_6}  presents the results of the semi-automatic coronary centerline extraction on the CTA datasets submitted to the Web-based platform. The following average results were obtained. A 84.3\% overlap with expert human manual annotations was achieved, until the first failure (OF) 65.3\%, in clinically relevant segments (radius $>$ 1.5 mm, OT) 84.4\%. In terms of accuraccy within the vessel (AI) an average of 0.41 mm where obtained. Generally, the method gets a 9$^{th}$ position in the overlaping rank and 12$^{th}$ position for the accuraccy rank in comparison with the CAT08 methods. Visual inspection revealed that most extraction errors occurred at the start of the vessel and near to the ostium.

\begin{table*}
\scriptsize
\caption{G\&T average overlap per dataset}
\centering
\begin{tabular}{|c|cc|cc|cc|}
\hline
\multicolumn{1}{|c|}{\textbf{Dataset}} &\multicolumn{2}{c|}{\textbf{OV}} &\multicolumn{2}{c|}{\textbf{OF}} &\multicolumn{2}{c|}{\textbf{OT}}\\
\multicolumn{1}{|c|}{\textbf{nr.}} &\multicolumn{1}{c|}{\textbf{\%}} &\multicolumn{1}{c|}{\textbf{score}} &\multicolumn{1}{c|}{\textbf{\%}} &\multicolumn{1}{c|}{\textbf{score}} &\multicolumn{1}{c|}{\textbf{\%}} &\multicolumn{1}{c|}{\textbf{score}}\\
\hline
0&91.5&46.7&64.6&38.5&91.5&46.7\\
1&100.0&100.0&100.0&100.0&100.0&100.0\\
2&98.2&61.9&88.7&58.3&98.1&61.8\\
3&71.1&56.7&70.6&67.4&71.1&52.9\\
4&91.3&69.1&91.7&69.2&93.5&72.1\\
5&92.9&50.6&44.9&22.5&92.9&46.6\\
6&81.6&65.8&84.0&67.0&83.2&79.1\\
7&48.2&25.2&49.5&25.7&48.2&24.3\\
\hline
\textbf{Avg.}&\textbf{84.4}&\textbf{57.8}&\textbf{73.8}&\textbf{54.2}&\textbf{84.9}&\textbf{59.4}\\
\hline
\end{tabular}
\vspace{-0.3cm}
\label{tb:tb_4_4}
\normalsize
\end{table*}

\begin{table*}
\scriptsize
\caption{G\&T average accuracy per dataset}
\centering
\begin{tabular}{|c|cc|}
\hline
\multicolumn{1}{|c|}{\textbf{Dataset}} &\multicolumn{2}{c|}{\textbf{AI}}\\
\multicolumn{1}{|c|}{\textbf{nr.}} &\multicolumn{1}{c|}{\textbf{mm}} &\multicolumn{1}{c|}{\textbf{score}}\\
\hline
0&0.45&30.6\\
1&0.46&28.9\\
2&0.40&25.3\\
3&0.46&32.2\\
4&0.35&25.8\\
5&0.40&37.1\\
6&0.35&29.5\\
7&0.39&32.0\\
\hline
\textbf{Avg.}&\textbf{0.40}&\textbf{30.1}\\
\hline
\end{tabular}
\vspace{-0.3cm}
\label{tb:tb_4_5}
\normalsize
\end{table*}

\begin{table*}
\scriptsize
\caption{G\&T Summary}
\centering
\begin{tabular}{|c|ccc|ccc|}
\hline
\multicolumn{1}{|c|}{\textbf{Measure}} &\multicolumn{3}{c|}{\textbf{\% / mm}} &\multicolumn{3}{c|}{\textbf{score}}  \\
\multicolumn{1}{|c|}{\textbf{}} &\multicolumn{1}{c|}{\textbf{min.}} &\multicolumn{1}{c|}{\textbf{max.}} &\multicolumn{1}{c|}{\textbf{avg.}} &\multicolumn{1}{c|}{\textbf{min.}} &\multicolumn{1}{c|}{\textbf{max.}} &\multicolumn{1}{c|}{\textbf{avg.}}\\
\hline
OV&13.9\%&100.0\%&84.4\%& 9.0&100.0&57.8\\
OF& 9.6\%&100.0\%&73.8\%& 4.8&100.0&54.2\\
OT&13.9\%&100.0\%&84.9\%& 9.0&100.0&59.4\\
AI&0.27 mm&0.52 mm&0.40 mm&22.5&43.1&30.1\\
\hline
\end{tabular}
\vspace{-0.3cm}
\label{tb:tb_4_6}
\normalsize
\end{table*}

METER TABLA DE MFLUX VS GULSUN

\begin{table*}
\scriptsize
\caption{Quantitative vesselness comparison using CAT08 framework}
\centering
\begin{tabular}{|c|cc|cc|cc|cc|}
\hline
\multicolumn{1}{|c|}{\textbf{Vesselness}} &\multicolumn{2}{c|}{\textbf{OV}} &\multicolumn{2}{c|}{\textbf{OF}} &\multicolumn{2}{c|}{\textbf{OT}}&\multicolumn{2}{c|}{\textbf{AI}}\\
\multicolumn{1}{|c|}{\textbf{Function}} &\multicolumn{1}{c|}{\textbf{\%}} &\multicolumn{1}{c|}{\textbf{score}} &\multicolumn{1}{c|}{\textbf{\%}} &\multicolumn{1}{c|}{\textbf{score}} &\multicolumn{1}{c|}{\textbf{\%}} &\multicolumn{1}{c|}{\textbf{score}}&\multicolumn{1}{c|}{\textbf{mm}} &\multicolumn{1}{c|}{\textbf{score}}\\
\hline
MFlux&84.3&60.4&65.3&50.7&84.4&59.2&0.41&29.2\\
G\&T&84.4&57.8&73.8&54.2&84.9&59.4&0.40&30.1\\
\hline
\end{tabular}
\vspace{-0.3cm}
\label{tb:tb_4_4}
\normalsize
\end{table*}


\subsection{Experiment III: MFlux in MICCAI2012 Challenge}

only overlapping measures are taking into account due to 


\section{Supervised Learning Experiments in Lesion Detection}


\subsection{Experiment I: RUSBoost in vessel hyphotesis}
\subsection{Experiment II: RUSBoost in segment hyphotesis}

\subsection{Experiment III: RUSBoost VS Random Forest}



