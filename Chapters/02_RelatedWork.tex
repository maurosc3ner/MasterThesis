\chapter{Related Work}\label{rel:rel}

\section{Centerline Extraction}

In recent years, the number (CAD) systems related to coronary diagnosis has increased \citep{Goldenberg2012}. Most, of the work that is done by these systems, involves a coronary tree extraction. Many axis extraction methods have been evaluated in \citep{Schaap2009} giving high accuracy compared to expert annotations. Here we give an overview of the methods evaluated according to similar techniques, we refer readers to the table \ref{tb:rel_cent} based on Schaap's work for an extensive evaluation and review on extraction methods according to the prior information and interaction levels (more than one point per vessel, one point per vessel, fully automatic).
\citep{Zhang2008, Tek2008, Metz2009, Krissian2008, Szymczak2008, Dikici2008, Friman2008} are vesselness-methods based or make use of a shortest path approach. \citep{Kitslaar2008, Castro2008} are morphology-based approaches. Inertia analysis method was developed by \citep{HernandezHoyos2008}. \citep{Wang2008} uses skeletonization and fuzzy conectedness tree algorithm. Finally, shape and appearance fitting models are used by \citep{Zambal2008}.
Minimum cost path methods that enhance specific structures of interest give better results \citep{Schaap2009}. The first group (shortest path-based methods) finds the coronary tree using one or more points and outperform the other ones.

\begin{table*}
\scriptsize
\caption{Overview of the previously published centerline extraction challenge \citep{Schaap2009}}
\centering
\begin{tabular}{|p{4cm}|p{10cm}|}
\hline
\multicolumn{1}{|p{4cm}|}{} &\multicolumn{1}{p{10cm}|}{\textbf{Description}}\\
\hline
\begin{flushleft}
Minimal user-interaction (More than one point per vessel)
\end{flushleft} & 
\begin{itemize}
	\item Local hessian-based vesselness (Zhang et al.). 
	\item Intensity threshold $+$ inertia analysis (Hern\'andez et al.).
	\item Multiple hypothesis tracking $+$ Fast marching (Friman et al.).
	\item Center of intensity plateaus in 2D slices (Szymczak et al.).
	\item Hessian based vesselness $+$ Minimum cost path (Metz et al..).
\end{itemize}\\
\hline
\begin{flushleft}
Semi-automatic (One point per vessel)
\end{flushleft} &
\begin{itemize}
	\item Hough-like $+$ minimum cost path (Dikici et al.).
	\item Iterative 3D morphology (Castro et al.).
	\item Feature space and vesselness $+$ Minimum cost path (Krissian et al.)
\end{itemize}\\
\hline
\begin{flushleft}
Full automatic
\end{flushleft} &
\begin{itemize}
	\item Multi-scale medialness based (Tek et al.).
	\item Segmentation and connected components (Kitslaar et al.).
	\item Shape and appearance fitting models (Zambal et al.).
	\item Fuzzy conectedness tree algorithm (Wang et al.).
\end{itemize}\\
\hline
\end{tabular}
\vspace{-0.3cm}
\label{tb:rel_cent}
\normalsize
\end{table*}

\section{Coronary Lesion Detection}

Here, we give an overview of learning-based methods for plaque detection in CTA images. We refer readers to  \citep{Kirisli2013} for an extensive evaluation and review on lesion detection.

Coronary lesions have no specific shape, size or position along the axis, they can show-up anywhere there is a vessel. An appropriate pattern should therefore be defined to capture the entire information of the lesion \citep{Zuluaga2011c} defined a 2D cross sectional pattern to estimate the lesion density. \citep{Tessmann2009} proposed a 3D multi-scale cylindrical pattern, where they extracted sample positions on a cylinder for feature extraction. Similar pattern was used by \citep{Mittal2010} but using steerable features. Finally, \citep{Cetin2012} proposed region based cylindrical with a vessel intensity approach.

In addition, small amount of illness labeled data versus no illness data add an extra difficulty in terms of feature pre-processing in the learning method.\citep{Chawla2002} proposed synthetic over-sampling in the feature space to balance datasets before a boosting classification. They propose an over-sampling approach in which the minority class is over-sampled by creating “synthetic” examples rather than by over-sampling with replacement. \citep{Seiffert2010} undersamples the majority class(es) to do the same but in faster and simple way.
In coronary lesions, \citep{Mittal2010} apply a synthetic interpolation of lesion slices in conjunction with a 3D sampling pattern for  calcified and non-calcified plaques. \citep{Zuluaga2011Thesis} adress this problem from the semi-supervised learning perspective, formuling as a density level detection problem with no labeled data available but with a 2D intensity metric. In the same direction \citep{Zuluaga2011a} has took into consideration only positive and unlabeled slices as a second semi-supervised methodology.
