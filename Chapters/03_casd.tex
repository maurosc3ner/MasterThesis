\chapter{Coronary Artery Stenoses Detectection}
%

\section{Introduction}

breve intro
mencionar el trabajo previo sin detalles
motivacion
nuestra contribucion
Due to the drawbacks above mentioned, a typical direction lead us to combine both ways. I mean, taking the best of sampling strategies with the weight capacity of boosting algorithms. Chawla et al. [4][5] extend this idea in the feature space. They propose an over-sampling approach in which the minority class is over-sampled by creating “synthetic” examples rather than by over-sampling with replacement. This approach has been succesfully proved in had written recognition applications [10] performing certain operations on real training data. In [10], operations like rotation and skew were natural ways to perturb the training data. Chawla et al. generate synthetic examples in a less application-specific manner, by operating in “feature space” rather than “data space”. This approach effectively forces the decision region of the minority class to become more general and robust to overfitting problems. Seiffert et al. [1] implement a compact way RUSBOOST to do the same with comparable results to Chawla et al. [5] SMOTEBOOST.

\section{Related work}

Coronary lesions have no specific shape, size or location along the centerline. The selected sampling pattern should therefore be invariant to such changes. [3]

The detection and quantification of vascular pathologies continue to be a tedious work for physicians who have to explore a vast amount of data using different visualization schemes based on advanced post-processing techniques. Recently, a family of methods that has made use of machine learning techniques has been proposed to perform this detection automatically. In addition, small amount of illness labeled data versus no illness data, make this problem enough suitable to an imbalanced data approach. Mittal et al. [CITAR A MITTAL] had showed that a synthetic interpolation of lesion slices to balance the dataset in conjunction with a 3D sampling pattern gives good results in calcified and non-calcified plaques. Zuluaga et al. [CITAR ZULUAGA] adress it from the semi-supervised learning perspective, formuling as a density level detection problem with no labeled data available but with a 2D intensity metric. Other aproach from the same authors [9] has took into consideration only positive and unlabeled slices in semi-supervised way. Taking into account all the things above, this problem is suitable for an imbalanced data formulation.



\section{Cylindrical Sampling Pattern}

Similar sampling pattern was also used in [3] only considering the calcified lesions. Tessmann et al. [2] proposed a multi-scale cylindrical pattern, where they extracted sample positions on a cylinder for feature extraction. Cetin et al. [4] proposed region based cylindrical approach, whereas, in our method, we extend the sampling-based features of [3], adding the radon analysis and an additional longitudinal analysis to capture the lesion as a whole. Further, since the lesions can potentially occur anywhere around the axis of the cylinder, we choose features that are rotation invariant about the axis.

\subsection{Intensity Features}

%We compute the following 9 features at a radius r, where (0 $<$ r $≤$ R) average, minimum and maximum intensities (I_avg,I_min,I_max), gradients along the radial direction (Gr_avg,Gr_min,Gr_max) and gradients along the tangent direction (Gtav,Gtmin,Gtmax) as [3]. Also, we add a simple radon analysis, giving us a total of 36 features per height of the cylinder. Therefore, with L = 5 and R = 3, we get a 5×3×12 = 188 dimensional feature vector. 
However, most of the local features are particularly significant for calcified lesions [2], as these regions have a high attenuation value. In order to capture the reduced contrast values originating from soft-plaques, longitudinal features are computed through elongated sampled feature pattern. 
Such as mean intensity, min and max, are used (see Fig. 2). We get 180+3x(36 cylindrical pattern)=288 features in total.

(Figure \ref{fig:2DMultiScaleIntensity})

\begin{figure}[ht]
	\centering
		%\includegraphics[width=3.0in]{./Figures/RadiusEstimation.png}
		\includegraphics[width=0.9\textwidth]{./Figures/2DMultiScaleIntensity.png}
	\caption[]{ (source: ).}
	\label{fig:2DMultiScaleIntensity}
\end{figure}


\subsection{Radon Transform Analysis}

The Radon transform is widely used in image processing for handling medical images [5]. The algorithm computes line integrals along many parallel beams or paths in an image from different angles theta by rotating the image around its centre. This transforms the image pixel intensity values along these lines into points in the Radon domain. Assuming that we are in the centerline, and the 2D view of the vessel is sane. If the vessel is healthy, the radon image will hold a tubular pattern with low mean values. In addition, This mean measure is shift-invariant, giving to the radon transform good possibilities to be included in our cylindrical pattern at different radius (See Fig. 4). In this way, our cylindrical pattern at every height l, becomes less sensitive to lesion position around the vessel’s axis.

\subsection{Longitudinal Analysis}

\subsection{Distance to Ostium}

\section{Learning}

\subsection{Dealing with skewed datasets}

When examples of one class greatly outnumber examples of the other class(es), traditional data mining algorithms tend to favor classifying examples as belonging to the overrepresented (majority) class. Such a model would be ineffective at identifying examples of the minority class, which is frequently the class of interest (the positive class) [1].

In general, there are 4 ways of dealing with skewed data:

Adjusting class prior probabilities to reflect realistic proportions.
Adjusting misclassification costs to represent realistic penalties.
Oversampling the minority class.
Undersampling the majority class.
Overall, the methods aiming to tackle with the imbalance data problem can be divided into two big categories by [2]:
Algorithm Specific Approach
Pre-processing for the data (sampling strategies)

Algorithm strategies

In the algorithmic side, boosting has been referred to as a kind of advanced data sampling technique. Boosting can improve the performance of any weak classifier (regardless of whether the data are imbalanced). The most common boosting algorithm is AdaBoost [3], which iteratively builds an ensemble of models. During each iteration, example weights are modified with the goal of correctly classifying examples in the next iteration, which were incorrectly classified during the current iteration.

Pre-processing strategies

In pre-processing strategies, data sampling balances the class distribution in the training data by either adding examples to the minority class (oversampling) [4][5] or removing examples from the majority class (undersampling). Both undersampling and oversampling have their benefits and drawbacks. The main drawback associated with undersampling is the loss of information that comes with deleting examples from the training data. On the other hand, oversampling results in no lost information but it can lead to overfitting due to the duplicated data.

\subsection{Supervised methods}


As an ideal case, we expect that lesions occur next to bifurcations, but, atherosclerosis happens everywhere there is a vessel. Therefore, training a learning algorithm with our recent first hypothesis could lead us to an overfitting scenario. Mittal et al. [1] and Zuluaga et al. [2] evidence problems at vessel bifurcartions. Mittal reduce its vessel scope to the three main vessels (RCA, LAD, LCX), this will avoid lesions repetition at the training stage suppressing the overfiting.

\subsection{RUSBoost: Hybrid}

\subsection{Random Forest}

\subsection{Parameter optimization}


\section{REFERENCIAS DEL CAPITULO}

[1] C. Seiffert, T.M. Khoshgoftaar, J. Van Hulse and A. Napolitano, "RUSBoost: A Hybrid Approach to Alleviating Class Imbalance, IEEE Transaction on Systems, Man and Cybernetics-Part A: Systems and Human, Vol.40(1), January 2010.

[2] Qi Y., "A Brief Literature Review of Class Imbalanced Problem", Carnegie Mellon University, 2004.

[3] Y.FreundandR.Schapire,“Experimentswithanewboostingalgorithm,” in Proc. 13th Int. Conf. Mach. Learn., 1996, pp. 148–156.

[4] Chawla, N., Bowyer, K., Hall, L., and Kegelmeyer, W. (2002) Smote: Synthetic minority over-sampling technique. Journal of Artificial Intelligence Research, 16, 321–357.

[5] Chawla, N., Lazarevic, A., Hall, L., and Bowyer, K. (2003) Smoteboost: improving prediction of the minority class in boosting, in VIIth European Conference on Principles and Practice of Knowledge Discovery in Databases(PKDD ´ 03), Lecture Notes on Computer Science, vol. 2838, Springer-Verlag, Lecture Notes on Computer Science, vol. 2838, pp. 107–119.

[7] Sushil Mittal, Yefeng Zheng, Bogdan Georgescu, Fernando Vega Higuera, Shaohua Kevin Zhou, Peter Meer, Dorin Comaniciu, "Fast Automatic Detection of Calcified Coronary Lesions in 3D Cardiac CT Images". MLMI 2010

[8] Maria A. Zuluaga, Isabelle E. Magnin, Marcela Hernández Hoyos, Edgar J. F. Delgado Leyton, Fernando Lozano, Maciej Orkisz: Automatic detection of abnormal vascular cross-sections based on density level detection and support vector machines. Int. J. Computer Assisted Radiology and Surgery 6(2): 163-174 (2011)

[9] Maria A. Zuluaga, Hush D., Edgar J. F. Delgado Leyton, Marcela Hernández Hoyos, Maciej Orkisz, "Learning from Only Positive and Unlabeled Data to Detect Lesions in Vascular CT Images"

%[10] Ha, T. M., & Bunke, H. "Off-line, Handwritten Numeral Recognition by Perturbation Method", Pattern Analysis and Machine Intelligence, 19/5, 535–539.
