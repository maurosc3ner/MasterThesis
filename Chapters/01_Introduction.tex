\chapter{Introduction}\label{intr:intr}

The work of this internship at CREATIS-UMR is the continuation of my first master year at uniandes. The topic of this tesis was carried out in the context of the "Stenoses Detection, Quantification and Lumen Segmentation" challenge\footnote{Challenges objectives, rules, and results can be visited in \href{http://www.grand-challenge.org}{http://www.grand-challenge.org}.}. It was held during the 15th International Conference on Medical Image Computing and Computer Assisted Intervention (\textit{MICCAI}), Nice, France, 1-6 October 2012.  Novel algorithms and visualization tools were to be developed, to support diagnosis in cardiovascular disease. The goal of the work described in this thesis is to develop and evaluate techniques to detect calcified plaque in coronary arteries from CTA. Since our previous MICCAI algorithm failed in some vessels extraction, our research has also pointed to the development of a vesselness-based framework for a correct centerline extraction. We have evaluated how these vesselness influence the accuracy of the centerline extracted in CTA images.

This document is organized as follows: This chapter presents the context, challenges in lesion assesment and an overview of the solution proposed. Chapter \ref{rel:rel} introduces state of the art related to centerline extraction and lesion detection. Chapter \ref{cent:cent} describes the proposed centerline framework and all the necessary steps required to do a centerline extraction. Chapter \ref{dect:dect} presents the supervised learning methodology and the feature extraction computed along the vessels extracted. Chapter \ref{eval:eval} gives an overview of the evaluation protocol followed in the different experiments. Chapter \ref{res:res} presents the experiments results. Finally, the chapter \ref{conc:conc} discuss the results and conclusions in terms of the project objetives. Two appendices are devoted to the computational tools developed.

\section{Computer-aided Diagnosis}
Cardiovascular diseases (\textit{CVD}) are a major cause of death worldwide. Due to a very high morbidity and mortality rates, they are classified as a major health problem over respiratory and digestive diseases, cancers, and others  ~\citep{WHO2011}.

The number of imaging studies to address this problem and the exponential growth of computational power explain the interest to develop computer aided systems for medical diagnosis purposes \citep{Ginneken2011}.

Today, computer-aided diagnosis (\textit{CAD}) systems have been introduced as complementary tools to assist radiologists, focusing their attention on certain image areas where potential lesions could be placed. Interpreting a medical image is extremely challenging, anatomical structures, noise and artifacts, make the image complicated. Therefore CAD systems serve as decision support tool for doctors in the diagnosis task. The doctor (\textit{generally a radiologist}) is always responsible for the final interpretation of the entire image. 

Recent studies have demonstrated that the use of advanced vessel analysis software improves the assessment of vascular diseases and reduces the inter-observer variability~\citep{DiCarli2007, Biermann2012}.

\subsection{How It Works?}

In general, three main steps can be distinguished in the literature on CAD systems \citep{Ginneken2001}: 1) pre-processing techniques; 2) segmentation of anatomical structures; and 3) analysis usually aimed to detect a specific kind of abnormality. In pre-processing, calibration, resampling and noise removal are usually performed on scanned images. The anatomic regions division of an image is done in the segmentation step. Finally, different measures (\textit{size, intensity, geometry, among others}) are done in the analyisis step to provide certain cues to doctors.

In heart disease, multiple CAD systems have been designed aiming to detect and quantify coronary issues. RCADIA \citep{Goldenberg2012} is a good example of a CAD system used as a triage planning tool\footnote{Triage planning involves developing and adopting a system or plan to prioritize patient treatment in particular contexts \citep{Iserson2007}} in clinical practice. Since the general process pipeline is similar through different medical fields, the literature \citep{Kirisli2013} agrees with two specific workflows for coronary lesion diagnosis (See Fig. \ref{fig:cad_wf}).
used in triage planning 



\begin{figure}[htbp]
	\centering
		\includegraphics[width=0.4\textwidth]{./Figures/cad_lesion_workflow.png}
	\caption[Coronary Lesion Detection Workflow]{Coronary Lesion Detection Workflow (source: \href{http://www.bigr.nl/publication/880}{Kirisli's Thesis}).}
	\label{fig:cad_wf}
\end{figure}

\subsection{Evaluation of CAD Systems}

Assesment can be very challenging and is considered a pinnacle of the computer-aided diagnosis and one of the most studied areas in medical image analysis. Quantitative measures (\textit{Sensitivity and specificity}) are only one way in the CAD systems evaluation spectrum \citep{Ginneken2011}. Sensitivity is determined by the percentage of positive cases well detected and placed by the CAD system. The number of false CAD marks per normal image or case is commonly used for specificity \citep{Castellino2005}. These measurements can be obtained by observing the performance of a CAD system on a set of ground truth cases. The ground truth cases are generally established by multiple radiologist consensus, adding inter-observer variation to the data. Stand alone sensitivity and specificity measurements are not enough to quantify algorithm performance, even, it becomes more difficult to prove a clinically relevant improvement by a certain method. 
Initiatives like the grand challenges in medical image analysis\footnote{Challenges objectives, rules, and results can be visited in \href{http://www.grand-challenge.org}{http://www.grand-challenge.org}.} address the evaluation challenge in medical imaging. It aims to create standard evaluation frameworks (\textit{tools and datasets}) to compare proposed algorithms with state of the art methods \citep{Hameeteman2011, Schaap2009, Kirisli2013} , but, it is still an open discussion.

%The remaining subsections introduce the medical context of the problem, the image acquisition techniques used for diagnosis, and the main challenges of the project. In Chapter 2, a revision of the most recent methods in these topics is presented. Chapter 3 and 4 summarize the methods. Chapter 5 and 6 presents quantitative and qualitative evaluation and results. Finally, the section 6 shows the conclussions of this work.

\section{The Coronary System}
%
The heart muscle (\textit{myocardium}) is an organ responsible for pumping blood throughout the vessels by repeated, rhythmic contractions (\textit{systole and diastole}). The correct functioning of the myocardium depends of the coronary arteries. The arteries supply oxygenated blood to the muscle, and can be grouped into two coronay trees starting from the aorta (\textit{called ostium}).The right coronary artery tree (\textit{RCA}) supplies oxygenated blood to the right ventricle, and also, to the inferior and infero-septal part of the left ventricle. The second tree (\textit{left coronary artery}) is divided in two branches: the left anterior descending artery (\textit{LAD}) supplies blood to the anterior region of the left ventricle, and the left circumflex artery (\textit{LCX}) that inject blood to the lateral and infero-lateral region of the left ventricle. The diameter of coronary arteries varies between 6{\,}mm at ostia locations and 0.5{\,}mm at 20{\,}cm from the starting point.

\begin{figure}[h]
	\centering
		\includegraphics[width=0.6\textwidth]{./Figures/Heart.png}
	\caption[Coronary Arteries Anatomy]{\textit{Left}, heart and main coronary arteries anatomy (source: \href{http://www.jeffersonhospital.org/
Tests-and-Treatments/coronary-artery-bypass-grafting.aspx}{Jefferson University Hospitals}).}
	\label{fig:He}
\end{figure}

Tipically, Cardiovascular diseases (\textit{CVD}) are due to the narrowing atherosclerosis formation by the development and accumulation of plaques of different materials (\textit{i.e. cholesterol, calcium, fibro-fatty deposits}) in artery walls. Once the atherosclerosis is present, the lumen blood flow is reduced yielding vessel necrosis and later heart attacks. Plaques can be grouped by their calcium score: 1) non-calcified plaques, having a lower density compared with the vessel lumen, 2) calcified plaques, presenting high density scores, and 3) mixed plaques that have non-calcified and calcified elements within a single plaque \citep{Pundziute2007}.

\begin{figure}[h]
	\centering
		\includegraphics[width=0.5\textwidth]{./Figures/Vessel.png}
	\caption[Cardiovascular Diseases]{Schematic illustration of the atherosclerosis formation (source:  \citep{Schaap2010Thesis}).}
	\label{fig:ves_plaq}
\end{figure}

\section{Conventional Coronary Angiography}

Conventional coronary angiography (\textit{CCA}) is the gold standard procedure to diagnose CVDs. In summary, a catheter is inserted through the femoral artery or the hand, then contrast agent is injected into the coronary arteries making them visible to X-ray imaging. The image (\textit{angiogram}) allows the radiologist detect possible vessel narrowing or plaque formations. However, this method is very invasive requiring high expertise to avoid complications (\textit{damages to internal arteries, alterations in pulse, etc.}), in addition, it requires a continuous flow of dose radiation and contrast agent to the patient to perform the diagnose.

\begin{figure}[htbp]
	\centering
		\includegraphics[width=0.9\textwidth]{./Figures/CCA_Hortense.png}
	\caption[Conventional Coronary Angiography]{Conventional Coronary Angiography (CCA) (a) Right coronary artery tree. (b) Left coronary artery tree  (source: \href{http://www.bigr.nl/publication/880}{Kirisli's Thesis}).}
	\label{fig:cca_im}
\end{figure}

\section{Computed Tomography Angiography}

A computerized tomography (\textit{CT}) coronary angiogram is an image modality to study the coronary arteries of the heart muscle with blood. Unlike a traditional coronary angiogram, CT angiograms don't use a catheter threaded through the coronary vessels. Instead, a coronary CT angiogram relies on intravenously injected iodine-based contrast fluid to highlight the coronary lumen. Then, a powerful X-ray (\textit{source and detector array}) machine rotate around the patient (See Fig. \ref{fig:ct_eq}). Yielding a detailed 3D human heart picture from the X-ray attenuation measured. Although CCA has better image resolution, its low but non-negligible risk of procedure due to its invasive nature \citep{Zanzonico2006} is always a drawback. Therefore, new CT angiograms (\textit{64-slice}) can be a better choice and it is expected to allow advanced computer-aided diagnosis like lumen morphology, reliable stenosis grading among others in the upcoming years.

\begin{figure}[htbp]
	\centering
		\includegraphics[width=0.5\textwidth]{./Figures/ct_equipment.jpg}
	\caption[Computed tomography equipment]{Computed tomography equipment (source: \href{http://www.radiologyinfo.org/}{RadiologyInfo.org}).}
	\label{fig:ct_eq}
\end{figure}

\section{Research Partners}
%
The project is part of a Franco-Colombian scientific collaboration developed by the following research teams: The Medical Imaging Research Center, CREATIS-UMR, Universit\'{e} Claude Bernard Lyon 1, Lyon, France; The Visual Computing Research Group, Imagine,Universidad de los Andes, Bogot\'{a}, Colombia and the Bio-informatics and Computer Graphics Group, Takina, Pontificia Universidad Javeriana, Bogot\'{a}, Colombia. In recent years, these institutions have permanently been working on carotid and coronary artery modeling in CTA images.

\begin{figure}[ht]
	\centering
		\includegraphics[width=0.5\textwidth]{./Figures/intContext.png}
	\caption[Research Team]{Research team and partners.}
	\label{fig:int_cont}
\end{figure}

Financial support for the projects of this collaboration has been provided by the ECOS Nord Commitee grant (\textit{C11S01}); the Administrative Department of Science, Technology and Innovation of Colombia - COLCIENCIAS and Uniandes Interfacultades (\textit{06-2010}). In Addition, this project is under the joint master degree program between the Electric Engineering Department of the University Claude Bernard Lyon 1 (\textit{UCBL}) and the Systems Engineering Department of the Universidad de los Andes (\textit{Uniandes}).

\section{Main Challenges and Solution Overview}
The following list outlines some of the challenges present in this master project.

\begin{itemize}
\item Avoid surrounding structures and image artifacts.
\item Extract accurate coronary arteries.
\item Manage arteries complexity and variability.
\item Identify calcified pathologies.
\item Visualize artery lesions.
\end{itemize}

\begin{figure}[htbp]
	\centering
		\includegraphics[width=0.9\textwidth]{./Figures/sol_over.png}
	\caption[Solution Workflow.]{Workflow for coronary lesion detection.}
	\label{fig:Workflow}
\end{figure}

The computational solution is defined as follows: First, a Dataset is loaded and two seed points are placed (\textit{ostia and distal locations}) by the radiologist in order to extract the coronary branch. Second, the coronary centerline is extracted. Next, a feature extraction pattern is computed along the centerline points. Finally, slice-based prediction is done to detect lesions along the centerline. (See Fig. \ref{fig:Workflow}).