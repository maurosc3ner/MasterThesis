\chapter{Introduction}
%

This document presents some contributions in two active research topics that helps to understand and quantify cardiovascular diseases (CVD): artery lumen segmentation and plaque detection. This study was mainly focused in coronary arteries due to the social, medical, and academic interest around the world in this subject due to a very high morbidity and mortality related to the coronary heart disease. 

The following subsections introduce the relevant medical context about arteries and their illnesses, the image acquisition techniques used for diagnosis, and the main challenges to extract relevant data. In Chapter 2, a revision of the most recent methods in these topics is presented. Chapter 3 and 4 summarize the methods used and their results. Finally, the conclusions of this work are collected in Chapter 5.

\section{Medical Context}
%

Cardiovascular diseases remain as the leading cause of death in the world. The World Health Organization (WHO) classifies them as the highest health problem with an alarming level (39\%) of deaths (17 millions in 2008) caused by non-communicable diseases among people under age of 70, followed by respiratory and digestive diseases (30\%), cancers (27\%), and others~\citep{WHO2011}. The most common reason of heart attacks and cerebrovascular problems is the blockage of the main arteries caused by the development of the atherosclerotic plaque, which begins by an accumulation of fatty materials in the inner arterial wall layer, reducing the blood flow to myocardium or to the brain.

In order to have a reliable and efficient diagnosis, new imaging techniques have been developed to help radiologists and physicians in early detection and treatment monitoring of CVDs. Nevertheless, these tasks can be tedious, time-consuming and error-prone, even for experienced specialists, requiring the use of robust methods to extract the most important information about diseases and anatomical characteristics. Recent studies have demonstrated that the use of advanced vessel analysis software improves the assessment of vascular diseases and reduces the inter-observer variability~\citep{DiCarli2007, Biermann2012}.

\subsection{Arterial Anatomy and Vascular Diseases}
%

Artery and vein wall has a layer structure composed by the \textit{tunica adventitia} (outermost layer), the \textit{tunica media} and the \textit{tunica intima} (innermost layer). The interior of the vessel enclosed by the \textit{tunica intima} is called the vessel lumen.

Coronary arteries supply the cardiac muscle (myocardium) with oxygenated blood. Two main coronary trees cover the left and right part of this muscle and they start at two locations called \textit{ostia} (Figure \ref{fig:HeartVessels}). The right coronary tree has a principal branch which is usually referred to as the right coronary artery (RCA). On the other side, the left coronary tree begins with a thicker branch called the left coronary artery (LCA), but not too far it divides into the left anterior descending artery (LAD) and the left circumflex artery (LCX). This second tree has an important influence in the correct work of the left ventricle that is responsible for providing all the rest of the body with oxygenated blood. The diameter of coronary arteries varies between 6{\,}mm at ostia locations and 0.5{\,}mm at 20{\,}cm from the starting point.

\begin{figure}[ht]
	\centering
		%\includegraphics[width=3.0in]{./Figures/RadiusEstimation.png}
		\includegraphics[width=0.9\textwidth]{./Figures/HeartVesselsSMALL.png}
	\caption[Coronary arteries anatomy and diseases]{\textit{Left}, heart and main coronary arteries anatomy (source: \href{http://www.jeffersonhospital.org/
Tests-and-Treatments/coronary-artery-bypass-grafting.aspx}{Jefferson University Hospitals}). \textit{Right}, schematic illustration of a vascular disease from \citep{Schaap2010Thesis}.}
	\label{fig:HeartVessels}
\end{figure}

%On the other hand, carotid arteries (left and right) transport blood to the head. In the lowest part, in the base of the neck, it could have a diameter of 6mm to 8mm. This first segment until the first bifurcation is called the common carotid artery. Then, two branches continue: the first one (external carotid artery) to the face and some external tissues, and the second one (internal carotid artery) supplies 80\% of the brain and the ocular orbits.

Commonly, CVDs are due to atherosclerosis characterized by the development and accumulation of plaques of different materials (i.e. cholesterol, calcium, fibro-fatty deposits) in artery walls reducing the blood flow. These plaques can be classified in three types: 1) non-calcified plaques, having a lower density compared with the vessel lumen, 2) calcified plaques, presenting high density scores, and 3) mixed plaques that have non-calcified and calcified elements within a single plaque~\citep{Pundziute2007}.

\subsection{Arteriography Vs. Computed Tomography Angiography }
%
Conventional invasive coronary angiography (arteriography) is still the standard procedure to diagnose CVDs. This method consists in the insertion of a catheter in a principal artery (usually in the leg or the forearm) until it arrives near the zone to be captured. At this point, a contrast agent is injected into the vessel and the zone is imaged using X-ray based techniques. The result is a projected image (angiogram) of arterial trees in the region of interest, obtained with a high refreshment rate. Looking at these images, physicians can detect possible diseases because of the reduction of lumen and blood flow. However, this method is very invasive requiring a high expertise to avoid complications (damages to internal arteries, alterations in pulse, etc.), it requires a continuous flow of radiation and contrast agent  injection, and sometimes the 2D projection hides narrowings that could be easily seen from another point of view.

\begin{figure}[htbp]
	\centering
		%\includegraphics[width=3.0in]{./Figures/RadiusEstimation.png}
		\includegraphics[width=0.9\textwidth]{./Figures/CPRWidgetSMALL.png}
	\caption[Visualization techniques for coronary arteries analysis.]{Left image presents 2D slices (axial, sagital, and coronal), center image is a volume rendering of the CTA, and right image presents CPR image with three orthogonal planes extracted at yellow, blue, and red locations.}
	\label{fig:CPRWidget}
\end{figure}

Computed tomography angiography (CTA) is a non-invasive technique that consists of 3D acquisitions of the area of study, enhancing arteries by using a contrast medium. The high resolution of modern scanners (64-slice CT scanner can achieve a resolution of $0.3 \times 0.3 \times 0.4${\,}mm$^3$ per voxel) permits not only to make an approximation of the lumen volume, but also to locate and quantify soft and calcified plaques. Calcified lesions can be spotted as small, bright regions while soft plaques are usually lesions with lower intensity values than the artery lumen. The most common methods to study and visualize these images are multi-planar reformation (MPR), curved planar reformation (CPR), and 3D rendering techniques such as volume rendering or maximum intensity projections. Both MPR and CPR correspond to a set of 2D slices extracted from the original image: sagital, coronal, and axial views in the case of MPR, and a set of continuous slices, orthogonal to the axis tangent at each point, in the case of CPR (Figure \ref{fig:CPRWidget} \footnote{This visualization results make part of the creaCoro project (\href{http://www.creatis.insa-lyon.fr/site/en/CreaCoro}{http://www.creatis.insa-lyon.fr/site/en/CreaCoro}).}). A more detailed revision about these techniques, their advantages and disadvantages can be found in \citep{Wang2011Thesis}.

Even if these new technologies improve the quality of the diagnosis and reduce the risks of invasive procedures, the amount of radiation used to obtain these images is considerably higher compared to the conventional methods. Moreover, the elevated quantity of data to evaluate due to 3D captures leads to other on-going challenges to detect clinically relevant problems.

\section{Main Challenges and Workflow}
%
Despite the possibility to improve the study of arteries thanks to high 3D resolution obtained by CTA, new difficulties appear from the image processing point of view. The following list proposed by Lesage \citep{Lesage2009Thesis} summarizes the main problems in this research area:

\begin{itemize}
	\item Size of data and resolution: CTA images typically reach sizes of $512\times512\times512$ voxels. In the case of coronaries, the radius varies between 1 to 10 voxels.	
	\item Acquisition issues: noise, partial volume effect, and artifacts influence the results.	
	\item Arteries complexity and variability: location, curvatures, sizes, and geometries are different in each patient.
	\item Pathologies and altering objects: presence of plaques, stents and bypasses affect the geometry and appearance.
	\item Surrounding structures: in the case of coronary trees, vein networks and heart chambers can be mistaken for arteries because of their similar appearance or shape. 
	%; in carotids, the yugular vein and bones can introduce the same problems.
\end{itemize}

\begin{figure}[htbp]
	\centering
		%\includegraphics[width=3.0in]{./Figures/RadiusEstimation.png}
		\includegraphics[width=0.9\textwidth]{./Figures/Workflow.png}
	\caption[Workflow for coronary arteries analysis.]{Workflow for coronary arteries analysis.}
	\label{fig:Workflow}
\end{figure}

Considering these issues, the analysis of arteries and their pathologies is usually pursued by following a workflow that begins with the localization of some starting seed (generally at \textit{ostia} locations) in order to initialize an axis extraction algorithm to identify the path of the main arteries. Additionally, the lumen segmentation can also be a result of this step or, depending on the approach used, it can give a first approach to the volume of interest to be processed. After these first steps, both lesion detection and stenosis quantification can be achieved in order to present specific and objective data about diseases to physicians (Figure \ref{fig:Workflow}).

\section{Institutional Context}
%

The work presented in this master thesis was carried out in the context of the ``Stenoses Detection, Quantification and Lumen Segmentation'' challenge. It was held during the 15th International Conference on Medical Image Computing and Computer Assisted Intervention (MICCAI), Nice, France, 1-6 October 2012. This challenge makes part of the ``Grand Challenges in Medical Image Analysis`` initiative\footnote{Challenges objectives, rules, and results can be visited in \href{http://www.grand-challenge.org}{http://www.grand-challenge.org}.} that aims to create standardize evaluation frameworks to compare novel algorithms in standart datasets.

The project was a collaboration of three biomedical imaging research groups at three universities in two countries (The Medical Imaging Research Center, CREATIS-UMR, Universit\'{e} Claude Bernard Lyon 1, Lyon, France; The Visual Computing Research Group, Imagine,Universidad de los Andes, Bogot\'{a}, Colombia; Bio-informatics and Computer Graphics Group, Takina, Pontificia Universidad Javeriana, Bogot\'{a}, Colombia). In Addition, this project is under the joint master degree program between the Electric Engineering Department of the University Claude Bernard Lyon 1 (UCBL) and the Systems Engineering Department of the Universidad de los Andes (Uniandes). 
The main work field of the research teams mentioned above is the signal processing and medical imaging with the goal of proposing innovative approaches in the pursuit of solutions for health issues. 