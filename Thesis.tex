%% ----------------------------------------------------------------
%% Thesis.tex -- MAIN FILE (the one that you compile with LaTeX)
%% ---------------------------------------------------------------- 

% Set up the document
\documentclass[a4paper, 11pt, oneside]{Thesis}  % Use the "Thesis" style, based on the ECS Thesis style by Steve Gunn
\graphicspath{{Figures/}}  % Location of the graphics files (set up for graphics to be in PDF format)

% Include any extra LaTeX packages required
%\usepackage[square, numbers, comma, sort&compress]{natbib}  % Use the "Natbib" style for the references in the Bibliography
\usepackage[comma, sort&compress]{natbib}  % Use the "Natbib" style for the references in the Bibliography
\usepackage{verbatim}  % Needed for the "comment" environment to make LaTeX comments
\usepackage{vector}  % Allows "\bvec{}" and "\buvec{}" for "blackboard" style bold vectors in maths
%\usepackage{algpseudocode}
\usepackage[titletoc]{appendix}
\usepackage{algorithm,algorithmic}
\usepackage{amssymb}
\usepackage{amsmath}
\hypersetup{urlcolor=blue, colorlinks=true}  % Colours hyperlinks in blue, but this can be distracting if there are many links.
%\usepackage{hyperref}
%% ----------------------------------------------------------------
\begin{document}
\frontmatter	  % Begin Roman style (i, ii, iii, iv...) page numbering

% Set up the Title Page
\title  {Semi-automatic Coronary Lesion Detection in Computed Tomography Images}
\authors  {\texorpdfstring
            {\href{mailto:Esteban.Correa@creatis.insa-lyon.fr}{Esteban Mauricio Correa Agudelo}}
            {Esteban Mauricio Correa Agudelo}
            }
\addresses  {\groupname\\\deptname\\\univname}  % Do not change this here, instead these must be set in the "Thesis.cls" file, please look through it instead
\date       {\today}
\subject    {}
\keywords   {}

\maketitle
%% ----------------------------------------------------------------

\setstretch{1.3}  % It is better to have smaller font and larger line spacing than the other way round

% Define the page headers using the FancyHdr package and set up for one-sided printing
\fancyhead{}  % Clears all page headers and footers
\rhead{\thepage}  % Sets the right side header to show the page number
\lhead{}  % Clears the left side page header

\pagestyle{fancy}  % Finally, use the "fancy" page style to implement the FancyHdr headers

%% ----------------------------------------------------------------
% Declaration Page required for the Thesis, your institution may give you a different text to place here
%\Declaration{
%
%\addtocontents{toc}{\vspace{1em}}  % Add a gap in the Contents, for aesthetics
%
%I, AUTHOR NAME, declare that this thesis titled, `THESIS TITLE' and the work presented in it are my own. I confirm that:
%
%\begin{itemize} 
%\item[\tiny{$\blacksquare$}] This work was done wholly or mainly while in candidature for a research degree at this University.
% 
%\item[\tiny{$\blacksquare$}] Where any part of this thesis has previously been submitted for a degree or any other qualification at this University or any other institution, this has been clearly stated.
% 
%\item[\tiny{$\blacksquare$}] Where I have consulted the published work of others, this is always clearly attributed.
% 
%\item[\tiny{$\blacksquare$}] Where I have quoted from the work of others, the source is always given. With the exception of such quotations, this thesis is entirely my own work.
% 
%\item[\tiny{$\blacksquare$}] I have acknowledged all main sources of help.
% 
%\item[\tiny{$\blacksquare$}] Where the thesis is based on work done by myself jointly with others, I have made clear exactly what was done by others and what I have contributed myself.
%\\
%\end{itemize}
% 
% 
%Signed:\\
%\rule[1em]{25em}{0.5pt}  % This prints a line for the signature
% 
%Date:\\
%\rule[1em]{25em}{0.5pt}  % This prints a line to write the date
%}
%\clearpage  % Declaration ended, now start a new page

%% ----------------------------------------------------------------
% The " Quote Page"
\pagestyle{empty}  % No headers or footers for the following pages

\null\vfill
% Now comes the " Quote", written in italics
\textit{``Brick walls are there for a reason... they let us prove how badly we want things''... Randy Pausch}
%\\
%\\
%--------------------
%
%\begin{tabbing}
%\textit{On}\= \textit{ce again}, \\ \> \textit{to my family, }\\ \> \textit{to my teachers,} \\ \> \textit{to my friends...}
%\end{tabbing}


\vfill\vfill\vfill\vfill\vfill\vfill\null
\clearpage  % Funny Quote page ended, start a new page
%% ----------------------------------------------------------------

%% The Abstract Page
%\addtotoc{Abstract}  % Add the "Abstract" page entry to the Contents
%\abstract{
%\addtocontents{toc}{\vspace{1em}}  % Add a gap in the Contents, for aesthetics
%
%%The Thesis Abstract is written here (and usually kept to just this page). The page is kept centered vertically so can expand into the blank space above the title too\ldots
%Abstract en ingles
%%Furthermore, coronary arteries play an important role in cardiovascular system and its diseases are mainly caused by the development and accumulation of plaques in artery walls. 
%
%}
%
%\clearpage  % Abstract ended, start a new page
%% ----------------------------------------------------------------
%
%% The Abstract Page
%\addtotoc{R\'{e}sum\'{e}}  % Add the "Abstract" page entry to the Contents
%\resume{
%\addtocontents{toc}{\vspace{1em}}  % Add a gap in the Contents, for aesthetics
%
%%The Thesis Abstract is written here (and usually kept to just this page). The page is kept centered vertically so can expand into the blank space above the title too\ldots
%
%Abstract en frances
%%Les maladies cardiovasculaires demeurent la premi\`{e}re cause de mortalit\'{e} dans le monde. Pour d\'{e}tecter et quantifier les l\'{e}sions dans les art\`{e}res coronaires, l'approche classique est de commencer par localiser de points de d\'{e}part des art\`{e}res (ostia) et d'extraire la ligne centrale (axe) du vaisseau. Ces donn\'{e}es sont n\'{e}cessaires pour ex\'{e}cuter la segmentation du lumen vasculaire et elles aident \`{a} la r\'{e}alisation des t\^{a}ches suivantes comme la d\'{e}tection de l\'{e}sions et la quantification de st\'{e}noses. En raison de l'int\'{e}r\^{e}t social, m\'{e}dical et acad\'{e}mique de ce sujet, ce m\'{e}moire pr\'{e}sente quelques contributions \`{a} l'\'{e}tude d'art\`{e}res coronaires dans des images d'angiographie par tomodensitom\'{e}trique, particuli\`{e}rement dans l'extraction des axes et la segmentation du lumen.
%
%%Dans un premier temps, nous d\'{e}crivons une m\'{e}thode d'extraction des axes d\'{e}velopp\'{e}e dans notre groupe et nous proposons un algorithme pour le recentrage des axes dans les plans orthogonaux \`{a} leurs directions principales. Apr\`{e}s ce traitement, les nouveaux axes ont sembl\'{e} \^{e}tre mieux centr\'{e}s notamment dans le cas des sections saines, avec n\'{e}anmoins des probl\`{e}mes lorsque les axes pr\'{e}sentent des plaques calcifi\'{e}es ou des bifurcations. Suite \`{a} cela, nous pr\'{e}sentons six m\'{e}thodes de segmentation du lumen cat\'{e}goris\'{e}es dans deux groupes principaux: techniques classiques et techniques de classification. Les meilleurs r\'{e}sultats qualitatifs ont \'{e}t\'{e} obtenus par des m\'{e}thodes bas\'{e}es sur des contours actifs et de la croissance de r\'{e}gion. L'\'{e}tude montre quelques difficult\'{e}s pour faire une validation ad\'{e}quate avec les donn\'{e}es de r\'{e}f\'{e}rence, ce qui est une des raisons majeures qui fait que l'on obtient un score de Dice tr\`{e}s bas en utilisant un algorithme des k-moyennes (66.48 \%) score comparable aux r\'{e}sultats obtenus par un seuillage classique (65.20 \%), Ceci est caus\'{e} par la pr\'{e}sence des r\'{e}gions faisant partie du lumen vasculaire e.g. une partie de l'aorte pr\`{e}s de l'ostium, mais n'\'{e}tant pas annot\'{e}es comme telles. Ces r\'{e}gions sont donc reconnues comme des faux positifs. 
%
%\clearpage

%Toutes les techniques de segmentation test\'{e}es pr\'{e}sentent des inconv\'{e}nients importants en raison de la difficult\'{e} \`{a} exprimer un mod\`{e}le g\'{e}om\'{e}trique explicite et un profil d'intensit\'{e} g\'{e}n\'{e}rique des art\`{e}res coronaires, particuli\`{e}rement dans le cas de bifurcations et de l\'{e}sions. Afin d'\'{e}viter cette mod\'{e}lisation explicite, nous avons utilis\'{e} la m\'{e}thode de s\'{e}parateurs \`{a} vaste marge, consid\'{e}r\'{e}e comme une technique de classification puissante, mais nous avons obtenu des r\'{e}sultats faibles de segmentation avec un score Dice en dessous de 50\% et une augmentation consid\'{e}rable de co{\^u}ts informatiques pendant les phases d'entrainement. Ceci est probablement d{\^u} \`{a} une s\'{e}lection de descripteurs inadapt\'{e}e pour d\'{e}crire le lumen. Finalement, des efforts suppl\'{e}mentaires doivent \^{e}tre faits pour d\'{e}finir un cadre d'\'{e}valuation complet qui permettrait la comparaison des m\'{e}thodes et l'analyse des art\`{e}res coronaires.

%}

%\clearpage  % Abstract ended, start a new page
%% ----------------------------------------------------------------

\setstretch{1.3}  % Reset the line-spacing to 1.3 for body text (if it has changed)

% The Acknowledgements page, for thanking everyone
\acknowledgements{
\addtocontents{toc}{\vspace{1em}}  % Add a gap in the Contents, for aesthetics

First of all, I would like to thank my advisors Prof. Marcela Hern\'andez, Prof. Leonardo Fl\'orez and Prof. Maciej Orkisz for giving me the opportunity to work and learn by their side during my internship at CREATIS-UMR. It has been a pleasure to work with you, despite the long, tiring and sometimes difficult journeys.

I owe my deepest gratitude to all those who kindly shared some of their knowledge with no other interest than making things better. Thanks a lot to William Romero for the valuable discussions at the coffee break. Leonardo for his suggestions on code implementation. A special word goes to Juan Diego for his invaluable help, his disposition to answer all my questions and his great enthusiasm. You represent what I consider science should be.

A big thank-you goes to my family for their unconditional support. My father, mother and my love Paola for their encouragement and patience in this journey. 

Gracias totales.

}
\clearpage  % End of the Acknowledgements
%% ----------------------------------------------------------------

\pagestyle{fancy}  %The page style headers have been "empty" all this time, now use the "fancy" headers as defined before to bring them back


%% ----------------------------------------------------------------
\lhead{\emph{Contents}}  % Set the left side page header to "Contents"
\tableofcontents  % Write out the Table of Contents

%% ----------------------------------------------------------------
\lhead{\emph{List of Figures}}  % Set the left side page header to "List if Figures"
\listoffigures  % Write out the List of Figures

%% ----------------------------------------------------------------
%\lhead{\emph{List of Tables}}  % Set the left side page header to "List of Tables"
%\listoftables  % Write out the List of Tables

%% ----------------------------------------------------------------

%\setstretch{1.5}  % Set the line spacing to 1.5, this makes the following tables easier to read
%\clearpage  % Start a new page
%\lhead{\emph{Abbreviations}}  % Set the left side page header to "Abbreviations"
%\listofsymbols{ll}  % Include a list of Abbreviations (a table of two columns)
%{
%% \textbf{Acronym} & \textbf{W}hat (it) \textbf{S}tands \textbf{F}or \\
%\textbf{LAH} & \textbf{L}ist \textbf{A}bbreviations \textbf{H}ere \\
%
%}

%% ----------------------------------------------------------------

%\clearpage  % Start a new page
%\lhead{\emph{Physical Constants}}  % Set the left side page header to "Physical Constants"
%\listofconstants{lrcl}  % Include a list of Physical Constants (a four column table)
%{
%% Constant Name & Symbol & = & Constant Value (with units) \\
%Speed of Light & $c$ & $=$ & $2.997\ 924\ 58\times10^{8}\ \mbox{ms}^{-\mbox{s}}$ (exact)\\
%
%}

%% ----------------------------------------------------------------

%\clearpage  %Start a new page
%\lhead{\emph{Symbols}}  % Set the left side page header to "Symbols"
%\listofnomenclature{lll}  % Include a list of Symbols (a three column table)
%{
%% symbol & name & unit \\
%$a$ & distance & m \\
%$P$ & power & W (Js$^{-1}$) \\
%& & \\ % Gap to separate the Roman symbols from the Greek
%$\omega$ & angular frequency & rads$^{-1}$ \\
%}

%% ----------------------------------------------------------------
% End of the pre-able, contents and lists of things
% Begin the Dedication page

%\setstretch{1.3}  % Return the line spacing back to 1.3
%
%\pagestyle{empty}  % Page style needs to be empty for this page
%\dedicatory{For/Dedicated to/To my\ldots}
%
%\addtocontents{toc}{\vspace{2em}}  % Add a gap in the Contents, for aesthetics


%% ----------------------------------------------------------------
\mainmatter	  % Begin normal, numeric (1,2,3...) page numbering
\pagestyle{fancy}  % Return the page headers back to the "fancy" style

% Include the chapters of the thesis, as separate files
% Just uncomment the lines as you write the chapters
\lhead{\emph{Introduction}}  
\chapter{Introduction}
%

This document presents some contributions in two active research topics that helps to understand and quantify cardiovascular diseases (CVD): artery lumen segmentation and plaque detection. This study was mainly focused in coronary arteries due to the social, medical, and academic interest around the world in this subject due to a very high morbidity and mortality related to the coronary heart disease. 

The following subsections introduce the relevant medical context about arteries and their illnesses, the image acquisition techniques used for diagnosis, and the main challenges to extract relevant data. In Chapter 2, a revision of the most recent methods in these topics is presented. Chapter 3 and 4 summarize the methods used and their results. Finally, the conclusions of this work are collected in Chapter 5.

\section{Medical Context}
%

Cardiovascular diseases remain as the leading cause of death in the world. The World Health Organization (WHO) classifies them as the highest health problem with an alarming level (39\%) of deaths (17 millions in 2008) caused by non-communicable diseases among people under age of 70, followed by respiratory and digestive diseases (30\%), cancers (27\%), and others~\citep{WHO2011}. The most common reason of heart attacks and cerebrovascular problems is the blockage of the main arteries caused by the development of the atherosclerotic plaque, which begins by an accumulation of fatty materials in the inner arterial wall layer, reducing the blood flow to myocardium or to the brain.

In order to have a reliable and efficient diagnosis, new imaging techniques have been developed to help radiologists and physicians in early detection and treatment monitoring of CVDs. Nevertheless, these tasks can be tedious, time-consuming and error-prone, even for experienced specialists, requiring the use of robust methods to extract the most important information about diseases and anatomical characteristics. Recent studies have demonstrated that the use of advanced vessel analysis software improves the assessment of vascular diseases and reduces the inter-observer variability~\citep{DiCarli2007, Biermann2012}.

\subsection{Arterial Anatomy and Vascular Diseases}
%

Artery and vein wall has a layer structure composed by the \textit{tunica adventitia} (outermost layer), the \textit{tunica media} and the \textit{tunica intima} (innermost layer). The interior of the vessel enclosed by the \textit{tunica intima} is called the vessel lumen.

Coronary arteries supply the cardiac muscle (myocardium) with oxygenated blood. Two main coronary trees cover the left and right part of this muscle and they start at two locations called \textit{ostia} (Figure \ref{fig:HeartVessels}). The right coronary tree has a principal branch which is usually referred to as the right coronary artery (RCA). On the other side, the left coronary tree begins with a thicker branch called the left coronary artery (LCA), but not too far it divides into the left anterior descending artery (LAD) and the left circumflex artery (LCX). This second tree has an important influence in the correct work of the left ventricle that is responsible for providing all the rest of the body with oxygenated blood. The diameter of coronary arteries varies between 6{\,}mm at ostia locations and 0.5{\,}mm at 20{\,}cm from the starting point.

\begin{figure}[ht]
	\centering
		%\includegraphics[width=3.0in]{./Figures/RadiusEstimation.png}
		\includegraphics[width=0.9\textwidth]{./Figures/HeartVesselsSMALL.png}
	\caption[Coronary arteries anatomy and diseases]{\textit{Left}, heart and main coronary arteries anatomy (source: \href{http://www.jeffersonhospital.org/
Tests-and-Treatments/coronary-artery-bypass-grafting.aspx}{Jefferson University Hospitals}). \textit{Right}, schematic illustration of a vascular disease from \citep{Schaap2010Thesis}.}
	\label{fig:HeartVessels}
\end{figure}

%On the other hand, carotid arteries (left and right) transport blood to the head. In the lowest part, in the base of the neck, it could have a diameter of 6mm to 8mm. This first segment until the first bifurcation is called the common carotid artery. Then, two branches continue: the first one (external carotid artery) to the face and some external tissues, and the second one (internal carotid artery) supplies 80\% of the brain and the ocular orbits.

Commonly, CVDs are due to atherosclerosis characterized by the development and accumulation of plaques of different materials (i.e. cholesterol, calcium, fibro-fatty deposits) in artery walls reducing the blood flow. These plaques can be classified in three types: 1) non-calcified plaques, having a lower density compared with the vessel lumen, 2) calcified plaques, presenting high density scores, and 3) mixed plaques that have non-calcified and calcified elements within a single plaque~\citep{Pundziute2007}.

\subsection{Arteriography Vs. Computed Tomography Angiography }
%
Conventional invasive coronary angiography (arteriography) is still the standard procedure to diagnose CVDs. This method consists in the insertion of a catheter in a principal artery (usually in the leg or the forearm) until it arrives near the zone to be captured. At this point, a contrast agent is injected into the vessel and the zone is imaged using X-ray based techniques. The result is a projected image (angiogram) of arterial trees in the region of interest, obtained with a high refreshment rate. Looking at these images, physicians can detect possible diseases because of the reduction of lumen and blood flow. However, this method is very invasive requiring a high expertise to avoid complications (damages to internal arteries, alterations in pulse, etc.), it requires a continuous flow of radiation and contrast agent  injection, and sometimes the 2D projection hides narrowings that could be easily seen from another point of view.

\begin{figure}[htbp]
	\centering
		%\includegraphics[width=3.0in]{./Figures/RadiusEstimation.png}
		\includegraphics[width=0.9\textwidth]{./Figures/CPRWidgetSMALL.png}
	\caption[Visualization techniques for coronary arteries analysis.]{Left image presents 2D slices (axial, sagital, and coronal), center image is a volume rendering of the CTA, and right image presents CPR image with three orthogonal planes extracted at yellow, blue, and red locations.}
	\label{fig:CPRWidget}
\end{figure}

Computed tomography angiography (CTA) is a non-invasive technique that consists of 3D acquisitions of the area of study, enhancing arteries by using a contrast medium. The high resolution of modern scanners (64-slice CT scanner can achieve a resolution of $0.3 \times 0.3 \times 0.4${\,}mm$^3$ per voxel) permits not only to make an approximation of the lumen volume, but also to locate and quantify soft and calcified plaques. Calcified lesions can be spotted as small, bright regions while soft plaques are usually lesions with lower intensity values than the artery lumen. The most common methods to study and visualize these images are multi-planar reformation (MPR), curved planar reformation (CPR), and 3D rendering techniques such as volume rendering or maximum intensity projections. Both MPR and CPR correspond to a set of 2D slices extracted from the original image: sagital, coronal, and axial views in the case of MPR, and a set of continuous slices, orthogonal to the axis tangent at each point, in the case of CPR (Figure \ref{fig:CPRWidget} \footnote{This visualization results make part of the creaCoro project (\href{http://www.creatis.insa-lyon.fr/site/en/CreaCoro}{http://www.creatis.insa-lyon.fr/site/en/CreaCoro}).}). A more detailed revision about these techniques, their advantages and disadvantages can be found in \citep{Wang2011Thesis}.

Even if these new technologies improve the quality of the diagnosis and reduce the risks of invasive procedures, the amount of radiation used to obtain these images is considerably higher compared to the conventional methods. Moreover, the elevated quantity of data to evaluate due to 3D captures leads to other on-going challenges to detect clinically relevant problems.

\section{Main Challenges and Workflow}
%
Despite the possibility to improve the study of arteries thanks to high 3D resolution obtained by CTA, new difficulties appear from the image processing point of view. The following list proposed by Lesage \citep{Lesage2009Thesis} summarizes the main problems in this research area:

\begin{itemize}
	\item Size of data and resolution: CTA images typically reach sizes of $512\times512\times512$ voxels. In the case of coronaries, the radius varies between 1 to 10 voxels.	
	\item Acquisition issues: noise, partial volume effect, and artifacts influence the results.	
	\item Arteries complexity and variability: location, curvatures, sizes, and geometries are different in each patient.
	\item Pathologies and altering objects: presence of plaques, stents and bypasses affect the geometry and appearance.
	\item Surrounding structures: in the case of coronary trees, vein networks and heart chambers can be mistaken for arteries because of their similar appearance or shape. 
	%; in carotids, the yugular vein and bones can introduce the same problems.
\end{itemize}

\begin{figure}[htbp]
	\centering
		%\includegraphics[width=3.0in]{./Figures/RadiusEstimation.png}
		\includegraphics[width=0.9\textwidth]{./Figures/Workflow.png}
	\caption[Workflow for coronary arteries analysis.]{Workflow for coronary arteries analysis.}
	\label{fig:Workflow}
\end{figure}

Considering these issues, the analysis of arteries and their pathologies is usually pursued by following a workflow that begins with the localization of some starting seed (generally at \textit{ostia} locations) in order to initialize an axis extraction algorithm to identify the path of the main arteries. Additionally, the lumen segmentation can also be a result of this step or, depending on the approach used, it can give a first approach to the volume of interest to be processed. After these first steps, both lesion detection and stenosis quantification can be achieved in order to present specific and objective data about diseases to physicians (Figure \ref{fig:Workflow}).

\section{Institutional Context}
%

The work presented in this master thesis was carried out in the context of the ``Stenoses Detection, Quantification and Lumen Segmentation'' challenge. It was held during the 15th International Conference on Medical Image Computing and Computer Assisted Intervention (MICCAI), Nice, France, 1-6 October 2012. This challenge makes part of the ``Grand Challenges in Medical Image Analysis`` initiative\footnote{Challenges objectives, rules, and results can be visited in \href{http://www.grand-challenge.org}{http://www.grand-challenge.org}.} that aims to create standardize evaluation frameworks to compare novel algorithms in standart datasets.

The project was a collaboration of three biomedical imaging research groups at three universities in two countries (The Medical Imaging Research Center, CREATIS-UMR, Universit\'{e} Claude Bernard Lyon 1, Lyon, France; The Visual Computing Research Group, Imagine,Universidad de los Andes, Bogot\'{a}, Colombia; Bio-informatics and Computer Graphics Group, Takina, Pontificia Universidad Javeriana, Bogot\'{a}, Colombia). In Addition, this project is under the joint master degree program between the Electric Engineering Department of the University Claude Bernard Lyon 1 (UCBL) and the Systems Engineering Department of the Universidad de los Andes (Uniandes). 
The main work field of the research teams mentioned above is the signal processing and medical imaging with the goal of proposing innovative approaches in the pursuit of solutions for health issues. 

 % Introduction

\lhead{\emph{Related Work}}  
\input{./Chapters/02_RelatedWork} % Introduction

\lhead{\emph{Semi-automatic Coronary Artery Centerline Extraction Framework}}  
\input{./Chapters/03_scacef} % Experimental Setup

\lhead{\emph{Coronary Artery Stenoses Detection}}  
\chapter{Coronary Artery Stenoses Detection}\label{dect:dect}
%

\section{Introduction}

Today, coronary artery lesion assesment is time consumming and prone to errors for physicians, who have to explore a vast amount of data. For this reason, computer-aided diagnosis systems have been introduced as complementary tools to assist radiologists, focusing their attention to certain image areas where potential lesions could be placed.
A variety of learning-based algorithms have been proposed in the literature for plaque detection in CTAimages. Some authors \citep{Mittal2010, Zuluaga2011c} evidence problems dealing with artery bifurcations, due to their similarity to lesions. In order to avoid this problem, we take the best from both solutions. A vessel hypothesis is defined to keep the information at bifurcation zones, and a longitudinal analysis is added to our extended feature pattern, capturing the entire lesion and neighborhood information. Also, a simpler learning technique is used to reduce the training time related to oversample the minority class.
This method detect calcified lesions in coronary vessel centerlines, a steerable cylinder-based feature is computed along the coronary vessel, then, an adapted AdaBoost version for imbalanced problems \citep{Seiffert2010} is used in the prediction step. Finally, lesions visualization and quantitative measures are provided to assess the computer-aided diagnosis. Qualitative and quantitative evaluation shown good lesion detection in multi-vendor datasets.

The feature pattern is introduced in the following section. In Section \ref{dect:imba}, we discuse how to deal with the imbalanced nature of this dataset. Section \ref{dect:hypo} is dedicated to the hypothesis definition to avoid overfitting in the training step. Finally, the learning method is explained in the section \ref{dect:rus}

\section{Cylindrical Sampling Pattern}

Steerable features proposed by \citep{Zheng2007} are a sampling pattern methdology, where a specific shape of the pattern is selected and local features are sampled using oriented gradients along the pattern. Considering the vessel shape and appearance, a discrete cylinder pattern is well suited to compute the image features at every centerline point \ref{fig:cyl_sch}. The sampling points are selected around the centerline using an equi-angular discretization and three different radius scales. Since, this is not a multi scale cylinder, the length $L$ of the cylinder is fixed. Experimental results give us good performance using $L$ between $\left[5 ,9\right]$. 

\begin{figure}[ht]
	\centering
		\includegraphics[width=0.7\textwidth]{./Figures/cyl_sch.png}
	\caption[Cylinder Pattern]{Cylinder pattern along the centerline.}
	\label{fig:cyl_sch}
\end{figure}


\subsection{Intensity Features}

The following nine local image features are computed at a radius r, where (0 $<$ r $≤$ R) (See Fig. \ref{fig:2d_sp} ): average, minimum and maximum intensities ($I_{avg}$,$I_{min}$,$I_{max}$), gradients along the radial direction ($\nabla r_{avg}$,$\nabla r_{min}$,$\nabla r_{max}$) and gradients along the tangent direction ($\nabla t_{avg}$,$\nabla t_{min}$,$\nabla t_{max}$) as \citep{Mittal2010} (See Fig. \ref{fig:2DMultiScaleIntensity}).

\begin{figure}[ht]
	\centering
		\includegraphics[width=0.4\textwidth]{./Figures/2D_sp.png}
	\caption[Cylinder Pattern 2D View]{Cylinder pattern 2D view.}
	\label{fig:2d_sp}
\end{figure}

\begin{figure}[ht]
	\centering
		\includegraphics[width=0.6\textwidth]{./Figures/2DMultiScaleIntensity.png}
	\caption[Steerable Feature]{2D Steerable feature illustration in a cross-sectional slice.}
	\label{fig:2DMultiScaleIntensity}
\end{figure}

\subsection{Radon Transform Analysis}

The Radon transform is widely used in image processing for handling medical images \citep{Acharya2013}. The algorithm computes line integrals along many parallel beams or paths in an image from different angles theta by rotating the image around its centre. This transforms the image pixel intensity values along these lines into points in the Radon domain. Assuming that we are in the centerline, and the 2D view of the vessel is sane. If the vessel is healthy, the radon image will hold a tubular pattern with high mean values. In addition, This interpretation is shift-invariant, giving to the radon transform good possibilities to be included in our cylindrical pattern at different radius. In this way, our cylindrical pattern at every height $l$, becomes less sensitive to lesion position around the vessel’s axis (See Fig. \ref{fig:rd1}).

\begin{figure}[ht]
	\centering
		\includegraphics[width=0.7\textwidth]{./Figures/radon1.png}
		\includegraphics[width=0.7\textwidth]{./Figures/radon2.png}
	\caption[Radon Transform]{Radon transform and analyisis of some cross-sectional slices.}
	\label{fig:rd1}
\end{figure}

\subsection{Longitudinal Analysis}

Most of the cross-sectional features are particularly significant for calcified lesions \citep{Tessmann2009}, as these regions have a high attenuation value, but they are limited to a 2D approach. In order to capture these lesions in every cross-section, we extend the same approach (average, minimum and maximum) used above but computed through the elongated cylinder, adding robustness in consecutive slices with presence of calcification (See Fig. \ref{fig:longi_feat}). A total of $3 \times 36$ features were computed.

\begin{figure}[ht]
	\centering
		\includegraphics[width=0.4\textwidth]{./Figures/samp_pat.png}
	\caption[Elongated Cylinder Features]{Elongated cylinder features.}
	\label{fig:longi_feat}
\end{figure}

\subsection{Distance to Ostium}
A parametric curve is computed and normalized from the centerline, then the distance to ostium is added, since the intensity along the centerline varies a lot. 

Finally, with a cylinder height $L = 5$ and a radial sampling $R = 3$, we get a $188_{cs}+108_{long}+1_{dist}=289$ dimensional feature vector.

\section{Learning}

\subsection{Dealing with skewed datasets}\label{dect:imba}

Figure \ref{fig:skw_dat} shows the tabulate population of our dataset. The healthy slices outnumber calcified slices by 10 times at least. When examples of one class greatly outnumber examples of the other class(es), traditional data mining algorithms tend to favor classifying examples as belonging to the overrepresented (majority) class. Such a model would be ineffective at identifying examples of the minority class, which is frequently the class of interest (the positive class) \citep{Seiffert2010}.

\begin{figure}[ht]
	\centering
		\includegraphics[width=0.6\textwidth]{./Figures/skew_data.png}
	\caption[Imbalanced Data]{Imbalanced data at the feature matrix.}
	\label{fig:skw_dat}
\end{figure}


In general, there are 4 ways of dealing with skewed data \citep{Qi2004}:
\begin{itemize}
	\item Adjusting class prior probabilities to reflect realistic proportions.
	\item Adjusting misclassification costs to represent realistic penalties.
	\item Oversampling the minority class.
	\item Undersampling the majority class.
\end{itemize}

Overall, the methods aiming to tackle with the imbalance data problem can be divided into two big categories:
\begin{itemize}
	\item Algorithm specific approach.
	\item Pre-processing for the data (sampling strategies).
\end{itemize}

\textbf{Algorithm strategies}

In the algorithmic side, boosting has been referred to as a kind of advanced data sampling technique. Boosting can improve the performance of any weak classifier (regardless of whether the data are imbalanced). The most common boosting algorithm is AdaBoost \citep{Freund1996}, which iteratively builds an ensemble of models. During each iteration, example weights are modified with the goal of correctly classifying examples in the next iteration, which were incorrectly classified during the current iteration.

\textbf{Pre-processing strategies}

In pre-processing strategies, data sampling balances the class distribution in the training data by either adding examples to the minority class (oversampling) \citep{Chawla2002} or removing examples from the majority class (undersampling) \citep{Seiffert2010}. Both undersampling and oversampling have their benefits and drawbacks. The main drawback associated with undersampling is the loss of information that comes with deleting examples from the training data. On the other hand, oversampling results in no lost information but it can lead to overfitting due to the duplicated data.

\subsection{Vessel Hypothesis (CTA) VS Segment Hypothesis (CCA)}\label{dect:hypo}

Tipically, we expect that lesions occur next to bifurcations, but, atherosclerosis happens everywhere there is a vessel. \citep{Mittal2010} and~\citep{Zuluaga2011a} evidence problems at vessel bifurcartions. Mittal proposed a vessel approach reducing the training vessels to the three main vessels (RCA, LAD, LCX), whereas Zuluaga states the limitation of her cross-sectional feature and the posibility to extended it in a future work. Both of them impact directly into the algorithm overfitting and segment bifurcations issues. To tackle this problem, we take the best from both sides, We keep the vessel approach from Mittal et al. and train our algorithm with selected vessels to avoid lesion repetitions, also a 3D sampling pattern is applied to capture the bifurcation information as it is. It is important to remark that choosing a wrong hypothesis could lead us to an overfitting or not good generalization scenario in our approach.

\begin{figure}[h]
	\centering
		\includegraphics[width=0.5\textwidth]{./Figures/artery_anatomy.png}
	\caption[Axial Coronary Anatomy]{Axial coronary anatomy (source: \citep{Kirisli2013})}
	\label{fig:art_ana}
\end{figure}

\subsection{RUSBoost}\label{dect:rus}

RUSBoost is an algorithm to handle class imbalance problem in data with discrete class labels. It uses a combination of RUS (\textit{random under-sampling}) and the standard boosting procedure AdaBoost \citep{Freund1996}, to better model the minority class by removing majority class samples. RUSboost ~\citep{Seiffert2010} undersamples the majority class(es) for every weak learner in the ensemble  (\textit{decision tree, most usually}). For example, if the majority class has 10 times as many observations as the minority class, it is undersampled 1/10. If the ensemble has say 100 trees, every observation in the majority class is used 100/10=10 times by the ensemble on average. Every observation in the minority class is used 100 times, once for every tree. This combination of simplicity, speed (undersampling VS oversampling ) and the small amount of illness labeled data in the dataset (See Fig. \ref{fig:skw_dat}), makes RUSBoost a good choice for our problem. In a very simple manner, it works as:

\begin{algorithm}                      % enter the algorithm environment
\caption{RUSBoost pseudocode}          % give the algorithm a caption
\label{alg:rb}                           % and a label for \ref{} commands later in the document
\begin{algorithmic}                    % enter the algorithmic environment
	\STATE Given set S of examples $(x_{1},y_{1}),...,(x_{m},y_{m})$ with a minority class
	\STATE Initialize D as  AdaBoost, i.e. $D_{1}(i)=1/m$ for all $i$
    \FORALL{$each base learner$}
            \STATE Create a temporal set of data $S'_{t}$ with distribution $D'_{t}$ undersampling the majority class randomly
            \STATE Train a base learner with $S'_{t}$
            \STATE Test the base learner on all data and calculated the error
            \STATE Increase the importance of samples that are still miss-classified
    \ENDFOR
    \STATE return the final hypothesis $H(x)$
\end{algorithmic}
\end{algorithm} % Experimental Setup

\lhead{\emph{Evaluation}}  
\chapter{Evaluation}\label{eval:eval}

\section{Introduction}

Both quantitative and qualitative evaluations were conducted to assess the performance of the methods.In the case of centerline extraction, the quantitative evaluation provides an objective evaluation of the method and enables comparison with previously published methods. However, comparison with results in literature should always be done with care, as results have been obtained on different data sets. A limitation of quantitative evaluation is that it requires a set of manually annotated structures, which is time consuming and not feasible for large numbers of 3D data. Therefore, we also conducted qualitative evaluations, as they can be performed in less time and hence on a larger number of data. Quantitative evaluation measures used in this work are introduced in the section below (\ref{eval:Measures}).

\section{Materials}\label{eval:Materials}
 
The datasets were obtained from the publicly available Rotterdam Coronary Artery Algorithm Evaluation Framework\footnote{Rotterdam Coronary Artery Algorithm Evaluation Framework \href{http://coronary.bigr.nl/}{http://coronary.bigr.nl/}.}. A total of twenty-six multi-center multi-vendor datasets were used for this study. 
Experiments related to the centerline extraction methods use eight datasets (\textit{4 vessels}). Centerline points and lumen radius are provided. Three types of images are available to test the success rate of the methods (See table \ref{tb:eval_mat1}. Two problematic datasets of the MICCAI12 challenge are also tested. Voxel resolution in both sets was on average close to $0.3 \times 0.3 \times 0.4${\,}mm$^3$ and image dimensions of 512 x  512 x 400 voxels.

\begin{table*}
\scriptsize
\caption{Image quality of the CAT08 datasets and vessels}
\centering
\begin{tabular}{|c|c|c|c|c|}
\hline
\multicolumn{1}{|c|}{Image quality} &\multicolumn{1}{c|}{\textbf{Poor}} &\multicolumn{1}{c|}{\textbf{Moderate}} &\multicolumn{1}{c|}{\textbf{Good}}&\multicolumn{1}{c|}{\textbf{Total}}\\
\hline
\# of datasets&2 &3&3 &8\\
\# of vessels&8 &12&12 &32\\
\hline
\end{tabular}
\vspace{-0.3cm}
\label{tb:eval_mat1}
\normalsize
\end{table*}
 
The remaining 16 datasets are used for the lesion detection assesment. Complete annotations about plaques (\textit{plaque type, location and percentage of occlusion}) are also given for each image. A vessel separation has been done avoiding the lesion repetition in each branch. The vessel summary is provided (See table \ref{tb:eval_mat2}).

\begin{table*}
\scriptsize
\caption{Vessels summary}
\centering
\begin{tabular}{|c|c|c|}
\hline
\multicolumn{1}{|c|}{} &\multicolumn{1}{c|}{\textbf{Healthy}} &\multicolumn{1}{c|}{\textbf{Mixed or calcified}}\\
\hline
\# Training vessels &53 &27\\
\# Testing &33&13\\
\hline
Total &86&40 \\
\hline
\end{tabular}
\vspace{-0.3cm}
\label{tb:eval_mat2}
\normalsize
\end{table*}

For additional information about the image acquisition, data selection and reference standards, the reader may refer to ~\citep{Schaap2009} and Kirişli et al. ~\citep{Kirisli2013} articles, or to the framework website.

\section{Evaluation Measures}\label{eval:Measures}

\subsection{Centerline Extraction}
Four different measures (\textit{three for overlapping and one for accuraccy assesment}) are used in this section. All of them are based on the terms defined by \citep{Schaap2009} (See Fig. \ref{fig:exp3_ms}).

\begin{figure}[htbp]
	\centering
		\includegraphics[width=1.0\textwidth]{./Figures/eval_ms.png}
	\caption[Coronary Centerline Evaluation Measures CAT08]{Evaluation Measures used at CAT08 (source: \citep{Schaap2009}).}
	\label{fig:exp3_ms}
\end{figure}

From the overlap point of view, overlap \textbf{(OV)} represents the ability to track the complete vessel annotated by the human observers; overlap until first error \textbf{(OF)} determines how much of a coronary artery has been extracted before making the first error; Finally, overlap with the clinically relevant part of the vessel \textbf{(OT)} gives an indication how the method is able to track a clinical relevant vessel section. An algorithm score of 100 points is a perfect match between us and the reference, a score equal to 0 is a total failure in the extraction. In the accuraccy side, average inside \textbf{(AI)} is the average distance inside the vessel, it uses the annotated radius to score it. The observer score equal to 50 points, say if we do better ($>$50 points) or worst ($<$50 points).

\subsection{Lesion Detection}

For the machine learning section, the use of the confusion matrix provides us some insights about the algorithm performance. The calcification detection framework defined, use a slice-based  approach. Therefore, it is straightforward to use slices as the evaluating unit and propose slice-based definitions of TP, TN, FP and FN \ref{tb:les_measures}. 

\begin{table*}
\scriptsize
\caption{Stenosis detection, as compared to CTA and CCA reference standard. Descriptions of true-positive (TP), false-negative (FP), false-positive (FP) and true-negative (TN) detection.}
\centering
\begin{tabular}{|c|p{9cm}|}
\hline
\multicolumn{1}{|c|}{\textbf{Detection}} &\multicolumn{1}{p{9cm}|}{\textbf{Description} for slice-based analysis}\\
\hline
\textbf{TP}&Both the reference standard and the algorithmslice have mixed or calcified plaque.\\
\textbf{FN}&The reference standard slice has mixed or calcified plaque while the algorithm slice has not mixed or calcified plaque.\\
\textbf{TN}&The reference standard slice has not mixed or calcified plaque while the algorithm slice has mixed or calcified plaque.\\
\textbf{FP}&Both the reference standard and the algorithm slice have not mixed or calcified plaque.\\
\hline
\end{tabular}
\vspace{-0.3cm}
\label{tb:les_measures}
\normalsize
\end{table*}

Statistical measures \citep{Fawcett2004} like accuracy, specificity and precision were evaluated using the slice-based rules defined above.
\textbf{Sensitivity:} Equivalent to recall or true positive rate. The sensitivity tells us how likely the test is come back positive (there is not plaque) in someone who has not plaque.
\begin{center}
$Sens =  \cfrac{(TP)}{( TP + FN )} $
\end{center}

\textbf{Specificity:} Also called the true negative rate. The specificity tells us how likely the test is come back negative (there is plaque) in someone who has plaque.
\begin{center}
$Spec = \cfrac{TN}{FP+TN}$
\end{center}

\textbf{Positive predictive value:} The precision in non-calcified slices 
\begin{center}
$PPV= \cfrac{TP}{TP+FP}$
\end{center}

\textbf{Negative predictive value:} The precision in calcified slices.
\begin{center}
$PPV= \cfrac{TN}{FN+TN}$
\end{center}

\textbf{Accuracy:} The overall classification accuracy is defined as:
\begin{center}
$ACC= \cfrac{TP+TN}{TP+FP+FN+TN}$
\end{center}

\section{Centerline Extraction Experiments}

\subsection{Experiment I: MFlux in CAT08 Challenge}

The centerline experiments where done using the eight training datasets from CAT08 challenge. For each patient, four vessels are provided in separated files. Since the consensus centerlines (done by three observers) are available in this experiment with their inter-observer variability measures, a quantitative analysis was performed using the standard framework proposed by ~\citep{Schaap2009}. Also, dataset results have been sent to the Web-based evaluation framework in order to verify the method results.

\subsection{Experiment II: MFlux in MICCAI2012 Challenge}

During the ``Stenoses Detection, Quantification and Lumen Segmentation challenge'', our team failed extracting some  vessels in problematic images (dataset 01 and dataset 05). Therefore, a correct and complete extraction is mandatory with this new version. Figure \ref{fig:exp2_dt05} shows our new framework processing one of the problematic datasets (dataset 05).

\begin{figure}[htbp]
	\centering
		\includegraphics[width=4.0in]{./Figures/dt05v01.png}
		\includegraphics[width=4.0in]{./Figures/dt05v01_cpr.png}
	\caption[MICCAI2012 Extraction]{\textit{Top}, Interactive RCA vessel extraction (dataset 05) . \textit{Bottom}, CPR visualization of the vessel extracted.}
	\label{fig:exp2_dt05}
\end{figure}

\subsection{Experiment III: MFlux VS Gulsun \& Tek (itkRGC)}

In the third experiment, we investigate the robustness of our method against a new version of \citep{Tek2008} developed by our team at Creatis. The performance of the team implementation (G\&T) was evaluated on the same CAT08 datasets. An additional results submission was done to the Web-platform. Also, problematic datasets at MICCAI2012 were visually evaluated with the two implementations (MFlux, G\&T). Figure ~\ref{fig:exp3_cpr} shows an axe extracted by the methods against the reference.

\begin{figure}[htbp]
	\centering
		\includegraphics[width=4.0in]{./Figures/cpr_lkeb.png}
		\includegraphics[width=4.0in]{./Figures/cpr_gt.png}
		\includegraphics[width=4.0in]{./Figures/cpr_mflux.png}
	\caption[Centerline Comparison]{Centerline visual assesment between: \textit{Top}, MICCAI 2012 Reference. \textit{Middle}, G \& T axe. \textit{Bottom}, MFlux axe.}
	\label{fig:exp3_cpr}
\end{figure}

\section{Supervised Learning Experiment in Lesion Detection}

The experiment was a leave-one-out test, using the 16 data sets. A total of 126 vessels were used. 20 in the training step and remaining  have been used for quality assessment of the algorithm. Since, we are reducing the vessl scope (See section \ref{casd:hypo}),  no lesion repetions are present into the datasets. This experiment may not lead to outstanding results, but It will more realistic in terms of detection rates. Also, two additional algorithms (AdaBoostM1, Random Forest) have been added to check the RUSBoost performance with this skewed dataset. % Experimental Setup

\lhead{\emph{Results}}  
\chapter{Results}\label{res:res}

\section{Centerline Extraction Experiments}

\subsection{Experiment I: MFlux in CAT08 Challenge}

Tables \ref{tb:tb_4_1},\ref{tb:tb_4_2} and \ref{tb:tb_4_3}  presents the results of the semi-automatic coronary centerline extraction on the CTA datasets submitted to the Web-based platform. The following average results were obtained. 

\begin{table*}[h]
\scriptsize
\caption{MFlux average overlap per dataset}
\centering
\begin{tabular}{|c|cc|cc|cc|}
\hline
\multicolumn{1}{|c|}{\textbf{Dataset}} &\multicolumn{2}{c|}{\textbf{OV}} &\multicolumn{2}{c|}{\textbf{OF}} &\multicolumn{2}{c|}{\textbf{OT}}\\
\multicolumn{1}{|c|}{\textbf{nr.}} &\multicolumn{1}{c|}{\textbf{\%}} &\multicolumn{1}{c|}{\textbf{score}} &\multicolumn{1}{c|}{\textbf{\%}} &\multicolumn{1}{c|}{\textbf{score}} &\multicolumn{1}{c|}{\textbf{\%}} &\multicolumn{1}{c|}{\textbf{score}}\\
\hline
0&96.2&73.7&69.5&65.2&96.2&73.7\\
1&95.0&47.6&95.0&47.5&95.0&47.5\\
2&99.4&74.8&92.2&71.4&99.4&74.8\\
3&64.5&49.9&47.2&43.5&64.5&49.6\\
4&70.3&58.5&88.8&67.7&71.0&60.7\\
5&95.8&56.8&40.4&20.2&95.8&48.1\\
6&76.9&63.9&73.6&64.1&77.1&63.8\\
7&78.4&40.4&16.7&9.0&78.4&39.7\\
\hline
\textbf{Avg.}&\textbf{84.3}&\textbf{60.4}&\textbf{65.3}&\textbf{50.7}&\textbf{84.4}&\textbf{59.2}\\
\hline
\end{tabular}
\vspace{-0.3cm}
\label{tb:tb_4_1}
\normalsize
\end{table*}

\begin{table*}[h]
\scriptsize
\caption{MFlux average accuracy per dataset}
\centering
\begin{tabular}{|c|cc|}
\hline
\multicolumn{1}{|c|}{\textbf{Dataset}} &\multicolumn{2}{c|}{\textbf{AI}}\\
\multicolumn{1}{|c|}{\textbf{nr.}} &\multicolumn{1}{c|}{\textbf{mm}} &\multicolumn{1}{c|}{\textbf{score}}\\
\hline
0&0.43&33.8\\
1&0.42&30.6\\
2&0.38&27.6\\
3&0.49&29.6\\
4&0.35&27.3\\
5&0.43&34.5\\
6&0.37&27.2\\
7&0.43&24.3\\
\hline
\textbf{Avg.}&\textbf{0.41}&\textbf{29.2}\\
\hline
\end{tabular}
\vspace{-0.3cm}
\label{tb:tb_4_2}
\normalsize
\end{table*}

\begin{table*}[h]
\scriptsize
\caption{MFlux Summary}
\centering
\begin{tabular}{|c|ccc|ccc|}
\hline
\multicolumn{1}{|c|}{\textbf{Measure}} &\multicolumn{3}{c|}{\textbf{\% / mm}} &\multicolumn{3}{c|}{\textbf{score}}  \\
\multicolumn{1}{|c|}{\textbf{}} &\multicolumn{1}{c|}{\textbf{min.}} &\multicolumn{1}{c|}{\textbf{max.}} &\multicolumn{1}{c|}{\textbf{avg.}} &\multicolumn{1}{c|}{\textbf{min.}} &\multicolumn{1}{c|}{\textbf{max.}} &\multicolumn{1}{c|}{\textbf{avg.}}\\
\hline
OV& 8.5\%&100.0\%&84.3\%& 5.8&100.0&60.4\\
OF&10.1\%&100.0\%&65.3\%& 5.0&100.0&50.7\\
OT& 8.5\%&100.0\%&84.4\%& 5.4&100.0&59.2\\
AI&0.28 mm&0.56 mm&0.41 mm&21.7&41.3&29.2\\
\hline
\end{tabular}
\vspace{-0.3cm}
\label{tb:tb_4_3}
\normalsize
\end{table*}

A 84.3\% overlap with expert human manual annotations was achieved, until the first failure (OF) 65.3\%, in clinically relevant segments (radius $>$ 1.5 mm, OT) 84.4\%. In terms of accuraccy within the vessel (AI) an average of 0.41 mm where obtained. Generally, the method ranks ninth in the overlaping section and 12$^{th}$ for the accuraccy rank in comparison with the CAT08 methods. Visual inspection revealed that most extraction errors occurred at the start of the vessel and near to the ostium. 

\subsection{Experiment II: MFlux in MICCAI2012 Challenge}

Since MICCAI12 does not provide the lumen radius annotations, only overlapping measures are taking into account in this section. Although, the experiment goal was the correct extraction in the problematic datasets. An overall good overlapping was achieved. This could be due to an improvement in the quality of images in the challenge.

\begin{table*}[h]
\scriptsize
\caption{Dataset 05, vessel 0 (RCA) VS reference team KLEB (Manually Corrected)}
\centering
\begin{tabular}{|c|cc|cc|cc|}
\hline
\multicolumn{1}{|c|}{\textbf{Methods}} &\multicolumn{2}{c|}{\textbf{OV}} &\multicolumn{2}{c|}{\textbf{OF}} &\multicolumn{2}{c|}{\textbf{OT}}\\
\multicolumn{1}{|c|}{\textbf{}} &\multicolumn{1}{c|}{\textbf{\%}} &\multicolumn{1}{c|}{\textbf{score}} &\multicolumn{1}{c|}{\textbf{\%}} &\multicolumn{1}{c|}{\textbf{score}} &\multicolumn{1}{c|}{\textbf{\%}} &\multicolumn{1}{c|}{\textbf{score}}\\
\hline
G\&T (itkRGC)&99.0 &99.0 &99.0 &99.0&99.0&99.0\\
MFlux&98.67&74.83& 98.51&49.25& 98.50&71.64\\
\hline
\end{tabular}
\vspace{-0.3cm}
\label{tb:tb_4_7}
\normalsize
\end{table*}

\begin{table*}[h]
\scriptsize
\caption{Dataset 01, vessel 4 (L-PDA) VS reference team KLEB (Manually Corrected)}
\centering
\begin{tabular}{|c|cc|cc|cc|}
\hline
\multicolumn{1}{|c|}{\textbf{Methods}} &\multicolumn{2}{c|}{\textbf{OV}} &\multicolumn{2}{c|}{\textbf{OF}} &\multicolumn{2}{c|}{\textbf{OT}}\\
\multicolumn{1}{|c|}{\textbf{}} &\multicolumn{1}{c|}{\textbf{\%}} &\multicolumn{1}{c|}{\textbf{score}} &\multicolumn{1}{c|}{\textbf{\%}} &\multicolumn{1}{c|}{\textbf{score}} &\multicolumn{1}{c|}{\textbf{\%}} &\multicolumn{1}{c|}{\textbf{score}}\\
\hline
G\&T (itkRGC)&100 &100&100 &100&100&100\\
MFlux&100&100& 100&100&100&100\\
\hline
\end{tabular}
\vspace{-0.3cm}
\label{tb:tb_4_7}
\normalsize
\end{table*}
	

\subsection{Experiment III: MFlux VS G\&T (itkRGC)}

Tables \ref{tb:tb_4_4},\ref{tb:tb_4_5} and \ref{tb:tb_4_6} give the results of G\&T implementation on the same CTA datasets. Also, table \ref{tb:tb_4_7} resume the performance of both methods against the best and worst CAT08 methods and the team results for this challenge.

\begin{table*}[h]
\scriptsize
\caption{G\&T average overlap per dataset}
\centering
\begin{tabular}{|c|cc|cc|cc|}
\hline
\multicolumn{1}{|c|}{\textbf{Dataset}} &\multicolumn{2}{c|}{\textbf{OV}} &\multicolumn{2}{c|}{\textbf{OF}} &\multicolumn{2}{c|}{\textbf{OT}}\\
\multicolumn{1}{|c|}{\textbf{nr.}} &\multicolumn{1}{c|}{\textbf{\%}} &\multicolumn{1}{c|}{\textbf{score}} &\multicolumn{1}{c|}{\textbf{\%}} &\multicolumn{1}{c|}{\textbf{score}} &\multicolumn{1}{c|}{\textbf{\%}} &\multicolumn{1}{c|}{\textbf{score}}\\
\hline
0&91.5&46.7&64.6&38.5&91.5&46.7\\
1&100.0&100.0&100.0&100.0&100.0&100.0\\
2&98.2&61.9&88.7&58.3&98.1&61.8\\
3&71.1&56.7&70.6&67.4&71.1&52.9\\
4&91.3&69.1&91.7&69.2&93.5&72.1\\
5&92.9&50.6&44.9&22.5&92.9&46.6\\
6&81.6&65.8&84.0&67.0&83.2&79.1\\
7&48.2&25.2&49.5&25.7&48.2&24.3\\
\hline
\textbf{Avg.}&\textbf{84.4}&\textbf{57.8}&\textbf{73.8}&\textbf{54.2}&\textbf{84.9}&\textbf{59.4}\\
\hline
\end{tabular}
\vspace{-0.3cm}
\label{tb:tb_4_4}
\normalsize
\end{table*}

\begin{table*}[h]
\scriptsize
\caption{G\&T average accuracy per dataset}
\centering
\begin{tabular}{|c|cc|}
\hline
\multicolumn{1}{|c|}{\textbf{Dataset}} &\multicolumn{2}{c|}{\textbf{AI}}\\
\multicolumn{1}{|c|}{\textbf{nr.}} &\multicolumn{1}{c|}{\textbf{mm}} &\multicolumn{1}{c|}{\textbf{score}}\\
\hline
0&0.45&30.6\\
1&0.46&28.9\\
2&0.40&25.3\\
3&0.46&32.2\\
4&0.35&25.8\\
5&0.40&37.1\\
6&0.35&29.5\\
7&0.39&32.0\\
\hline
\textbf{Avg.}&\textbf{0.40}&\textbf{30.1}\\
\hline
\end{tabular}
\vspace{-0.3cm}
\label{tb:tb_4_5}
\normalsize
\end{table*}

\begin{table*}[h]
\scriptsize
\caption{G\&T Summary}
\centering
\begin{tabular}{|c|ccc|ccc|}
\hline
\multicolumn{1}{|c|}{\textbf{Measure}} &\multicolumn{3}{c|}{\textbf{\% / mm}} &\multicolumn{3}{c|}{\textbf{score}}  \\
\multicolumn{1}{|c|}{\textbf{}} &\multicolumn{1}{c|}{\textbf{min.}} &\multicolumn{1}{c|}{\textbf{max.}} &\multicolumn{1}{c|}{\textbf{avg.}} &\multicolumn{1}{c|}{\textbf{min.}} &\multicolumn{1}{c|}{\textbf{max.}} &\multicolumn{1}{c|}{\textbf{avg.}}\\
\hline
OV&13.9\%&100.0\%&84.4\%& 9.0&100.0&57.8\\
OF& 9.6\%&100.0\%&73.8\%& 4.8&100.0&54.2\\
OT&13.9\%&100.0\%&84.9\%& 9.0&100.0&59.4\\
AI&0.27 mm&0.52 mm&0.40 mm&22.5&43.1&30.1\\
\hline
\end{tabular}
\vspace{-0.3cm}
\label{tb:tb_4_6}
\normalsize
\end{table*}

\begin{table*}[h]
\scriptsize
\caption{Quantitative methods comparison using CAT08 framework}
\centering
\begin{tabular}{|c|cc|cc|cc|cc|}
\hline
\multicolumn{1}{|c|}{\textbf{Methods}} &\multicolumn{2}{c|}{\textbf{OV}} &\multicolumn{2}{c|}{\textbf{OF}} &\multicolumn{2}{c|}{\textbf{OT}}&\multicolumn{2}{c|}{\textbf{AI}}\\
\multicolumn{1}{|c|}{\textbf{}} &\multicolumn{1}{c|}{\textbf{\%}} &\multicolumn{1}{c|}{\textbf{score}} &\multicolumn{1}{c|}{\textbf{\%}} &\multicolumn{1}{c|}{\textbf{score}} &\multicolumn{1}{c|}{\textbf{\%}} &\multicolumn{1}{c|}{\textbf{score}}&\multicolumn{1}{c|}{\textbf{mm}} &\multicolumn{1}{c|}{\textbf{score}}\\
\hline
Friman et al.&99.3 &91.6& 94.6 &83.2 &99.5 & 90.0 &0.24 &47.9\\
G\&T (itkRGC)&84.4&57.8&73.8&54.2&84.9&59.4&0.40&30.1\\
MFlux&84.3&60.4&65.3&50.7&84.4&59.2&0.41&29.3\\
Hern\'andez et al. &77.0& 40.5 &52.1 &31.5 & 79.0 &45.3 &0.41 &29.3 \\
Castro et al.&72.6 &38.9 &45.6 &27.3 &73.8 & 41.0 &0.67 &21.2\\
\hline
\end{tabular}
\vspace{-0.3cm}
\label{tb:tb_4_7}
\normalsize
\end{table*}

Despite our limited comparison in terms of vesselness filters, our method is able to track centerline vessels with an accuracy comparable to the experts and with similar G\&T performance.

\section{Supervised Learning Experiment in Lesion Detection}

Table \ref{tb:ml_res} shows the average results of our method and the others for stenosis dectection measures: sensitivity, specificity, positive predictive value, negative predictive value and accuracy. In general, our results are
worse than the literature performance \citep{Mittal2010,Zuluaga2011a}(specificity $>$ 75\% on CTA). Therefore, the ability of our method to discriminate significant calcifications from non-significant ones remains  limited (66\%). Nevertheless, as compared to the non imbalanced options in the same dataset (AdaBoost M1 and Random Forest), our method leads the list in terms of specificity, over the testing set (See Fig \ref{fig:errcurve}).

\begin{figure}[ht]
	\centering
		\includegraphics[width=0.5\textwidth]{./Figures/289Final.png}
	\caption[Classification Curve]{(\textit{Top}) Classification error curve.}
	\label{fig:errcurve}
\end{figure}


\begin{table*}[h]
\scriptsize
\caption{Quantitative learning performance}
\centering
\begin{tabular}{|c|c|c|c|c|c|}
\hline
\multicolumn{1}{|c|}{\textbf{Methods}} &\multicolumn{1}{c|}{\textbf{Sens (\%)}} &\multicolumn{1}{c|}{\textbf{Spec (\%)}} &\multicolumn{1}{c|}{\textbf{PPV (\%)}}&\multicolumn{1}{c|}{\textbf{NPV (\%)}}&\multicolumn{1}{c|}{\textbf{Accu (\%)}}\\
\hline
AdaBoost M1&99.57 &59.55 &97.15 &90.99 &96.88 \\
\textbf{RUSBoost} &\textbf{98.84} &\textbf{66.29} &\textbf{97.59} &\textbf{80.55} &\textbf{96.65}\\
Random Forest&99.19 &52.81 &96.68 &82.46 &96.06\\
\hline
\end{tabular}
\vspace{-0.3cm}
\label{tb:ml_res}
\normalsize
\end{table*}

An additional qualitative evaluation (See Fig \ref{fig:res_vis}) reveals that we detect the majority of lesions, but from the slice-based point view, we are highly penalized in terms of false negatives, since radiologists assume a bigger lesion size (affecting more consecutive slices). Future work in a multi-scale cylinder could give better results.
 
\begin{figure}[ht]
	\centering
		\includegraphics[width=0.7\textwidth]{./Figures/visual_lesions.png}
	\caption[Visual Assesment in Lesion Detection]{Visual comparison between reference data (green=healthy, red=calcified) and results data (cyan curve peaks).}
	\label{fig:res_vis}
\end{figure} % Experimental Setup

\lhead{\emph{Discussion \& Conclusions}}  
\chapter{Conclusions}\label{conc:conc}

In this thesis we proposed and evaluated an algorithm to detect calcifications in CTA images. In addition, an interactive framework has been developed allowing the evaluation of different vesselness images with a minimum cost algorithm. Also, visualization tools are provided to highlight lesions to support radiologist's diagnosis. Both tools are available in the CreaTools environment\footnote{Medical image processing and visualization software and development tools, CreaTools \href{http://www.creatis.insa-lyon.fr/site/en/softwares-releases}{http://www.creatis.insa-lyon.fr/site/en/softwares-releases}}. In this chapter we summarize the main contributions of the work presented and discuss future research directions.

In Chapter 3 we proposed a flexible cost function to extract centerlines using vesselness flux images. Compared to other centerline extraction methods at CAT08 challenge, more optimization parameters are required to achieve outstanding performance. Moreover, better overlap and vessel accuraccy has been acomplished with this new framework compared to the old team solution. From the vesselness implementation side, further implementation improvements need to be done in order to be a practical solution in clinical procedures.

Chapter 4 presents the lesion detection algorithm. The method is developed using an adapted machine learning method to deal with the skewed dataset. Since not significant difference was founded between RUSBoost and AdaBoost. Supervised methods could be limitated in a CTA assesment approach. Nevertheless, Avoiding false positive bifurcation have been achieved, since we are using a vessel hypothesis and lesions occur next to bifurcations.

The next step towards automation would be the ostium detection for the centerline extraction algorithm, which is able to find the centerlines of all major coronary arteries from a standard clinical CTA dataset. 
Good results for automatic lesion detection extraction have been demonstrated with the results presented in Chapter 6. However, calcified detection of coronary lesions is still not a solved problem. Further research needs to be done in terms of the evaluation framework definition (False negative, etc). % Experimental Setup

%\input{./Chapters/Chapter7} % Conclusion

%% ----------------------------------------------------------------
% Now begin the Appendices, including them as separate files

\addtocontents{toc}{\vspace{2em}} % Add a gap in the Contents, for aesthetics
\begin{appendices}
\appendixpage
\noappendicestocpagenum
\addappheadtotoc
%\appendix % Cue to tell LaTeX that the following 'chapters' are Appendices
\lhead{Appendix A. \emph{frontAlgorithms: A Semi-automatic Coronary Centerline Extraction Framework}}
% Appendix A

\chapter{frontAlgorithms: A Semi-automatic Coronary Centerline Extraction Framework} \label{AppendixA}

\begin{figure}[ht]
	\centering
		\includegraphics[width=1.3\textwidth]{./Figures/extrac6.png}
	\label{fig:apa_cent1}
\end{figure}
\clearpage

\section{Copyright}
Esteban Correa, Universidad de los Andes or CREATIS-UMR do not offer any support for this product whatsoever. The program is offered free of charge. The executable program is copyrighted freeware by CREATIS-UMR.

\section{Contributors}

\textbf{Esteban Correa}
\href{mailto:esteban.correa@creatis.insa-lyon.fr }{esteban.correa@creatis.insa-lyon.fr }
Master Student for centerline extraction methods and development.
Universidad de los Andes, Bogot\'a D.C. Colombia.
Université Lyon 1, France.

\textbf{Leonardo Fl\'orez}
\href{mailto:florez-l@javeriana.edu.co }{florez-l@javeriana.edu.co }
Associate Professor and main software developer and centerline extraction methods.
TAKINA
Pontificia Universidad Javeriana, Bogot\'a D.C. Colombia.

\textbf{M. Orkisz}
\href{mailto:maciej.orkisz@creatis.insa-lyon.fr }{maciej.orkisz@creatis.insa-lyon.fr }
Associate Professor
CREATIS, CNRS UMR 5220, INSERM U1044.
Université Lyon 1, INSA Lyon. France.

\section{Introduction}
This software is a computer program whose purpose is to evaluate the performance of centerline extraction methods in the context of image processing (and more particularly on CTA images).
The software has been designed for two main purposes:

This method finds a coronary vessel centerline interactively. Two seed points are manually placed by the radiologist and the path is automatically extracted using a minimum cost path approach (Dijkstra’s algorithm). The cost to travel through a voxel is based on a generalized logistic function applied to the non-linear flux vesselness response \citep{Lesage2009a}  of the CTA image. Selective vessel radiuses can be provided to get rid off erroneously connected structures such coronary veins. Results shown robust and accurate centerline extraction in multi-vendor datasets.

Finally, the software allows you to compare the performance of vesselness in a shortest path approach. By default, two vesselness methods are provided \citep{Tek2008,Lesage2009a} with their respective cost functions. A graphical interface enables the algorithm progression.

The software is divided into 2 parts. We can identify:
\begin{itemize}
\item Seeds placement GUI.
\item Centerline Extraction Algorithm.
\end{itemize}

\subsection{System Requirements}

\begin{itemize}
\item At least 2Gb of ram (due to the size of CTA images).
\item Windows or Linux operating systems.
\item CreaTools Framework.
\end{itemize}

\subsection{CreaTools Package Requirements}

\begin{itemize}
\item itkRGC\/frontAlgorithms
\end{itemize}

\section{Point placement}

\textbf{File(s)}
\begin{itemize}
\item fa\_example\_extract\_path\_with\_dijkstra\_06.bbg
\end{itemize}

\textbf{Inputs}
\begin{itemize}
\item Image file (\* .mhd, \* .raw).
\end{itemize}

\textbf{Description}

\begin{figure}[ht]
	\centering
		\includegraphics[width=0.7\textwidth]{./Figures/apa_bbe.png}
	\caption[Radiologist's GUI]{(\textit{Top}) Start point (at ostium). (\textit{Bottom}) end point (distal location) (source: CreaTools screenshot).}
	\label{fig:apa_bbe}
\end{figure}

In order to integrate results from frontAlgorithms extraction with creaCoro aplication. We have build a new bbtk diagram to simulate in some way, how the radiologist works on the images (See Fig. \ref{fig:apa_bbe}). Basically, the user (radiologist) select start and end points of the coronary artery in CTA image (See Fig. \ref{fig:apa_sp1}). 

\begin{figure}[ht]
	\centering
		\includegraphics[width=0.8\textwidth]{./Figures/ShowNPoints1.png}
		\includegraphics[width=0.8\textwidth]{./Figures/ShowNPoints1.png}
	\caption[Radiologist's GUI]{(\textit{Top}) Start point (at ostium). (\textit{Bottom}) end point (distal location) (source: CreaTools screenshot).}
	\label{fig:apa_sp1}
\end{figure}

\section{Centerline Extraction}

\textbf{File(s)}
\begin{itemize}
\item fa\_example\_extract\_path\_with\_dijkstra\_06.bbg
\end{itemize}

\textbf{Inputs}
\begin{itemize}
\item Image file (\* .mhd, \* .raw).
\item Vesselness (offline computed) image file (\* .mhd, \* .raw) .
\item Seed point files (\* .txt)
\end{itemize}

\textbf{Outputs}
\begin{itemize}
\item vtk file containing the centerline parametric path.
\end{itemize}

\textbf{Description}

Currently, you can compute two types of vesselness images (Gulsun, MFlux), command line applications(package\/appli\/examples) are provided for this purpose. Once you have your vesselness image, you can set parameters of the cost function (See section \ref{cent:cost} in order to improve the extraction. Three pre-programmed functions are available. Finally you can launch the extraction method and see the progression interatively (See Fig. \ref{fig:apa_cent1}).

\begin{figure}[ht]
	\centering
		\includegraphics[width=0.5\textwidth]{./Figures/extrac1.png}
		\includegraphics[width=0.5\textwidth]{./Figures/extrac2.png}
		\includegraphics[width=0.5\textwidth]{./Figures/extrac3.png}
		\includegraphics[width=0.5\textwidth]{./Figures/extrac4.png}
	\caption[Centerline Extraction GUI 1]{Extraction progression (source: CreaTools screenshot).}
	\label{fig:apa_cent1}
\end{figure}

\begin{figure}[ht]
	\centering
		\includegraphics[width=0.6\textwidth]{./Figures/extrac6.png}
		\includegraphics[width=0.6\textwidth]{./Figures/extrac7.png}
		\includegraphics[width=0.6\textwidth]{./Figures/extrac8.png}
		\includegraphics[width=0.6\textwidth]{./Figures/extrac9.png}
	\caption[Centerline Extraction GUI 2]{Centerline extracted (source: CreaTools screenshot).}
	\label{fig:apa_cent1}
\end{figure}

	% Appendix Title

\lhead{Appendix B. \emph{creaCoro: A Coronary Lesion Visualization Framework}}
% Appendix A

\chapter{CreaCoro: A Coronary Lesion Visualization Framework}\label{AppendixB}

\begin{figure}[ht]
	\centering
		\includegraphics[width=1.0\textwidth]{./Figures/heart3d.png}
	\label{fig:longi_feat}
\end{figure}
\clearpage
\section{Copyright}
Esteban Correa, Universidad de los Andes or CREATIS-UMR do not offer any support for this product whatsoever. The program is offered free of charge. The executable program is copyrighted freeware by CREATIS-UMR.

\section{Contributors}

\textbf{Esteban Correa}
\href{mailto:esteban.correa@creatis.insa-lyon.fr }{esteban.correa@creatis.insa-lyon.fr }.
Master Student for methods and development.
Universidad de los Andes, Bogot\'a D.C. Colombia.
Université Lyon 1, France.

\textbf{Claire Mouton}
\href{mailto:claire.mouton@creatis.insa-lyon.fr}{claire.mouton@creatis.insa-lyon.fr}.
Software engineer, info-dev team.
CREATIS, CNRS UMR 5220, INSERM U1044.
Université Lyon 1, INSA Lyon. France.

\textbf{Eduardo Enrique Davila Serrano}
\href{mailto:eduardo.davila@creatis.insa-lyon.fr }{eduardo.davila@creatis.insa-lyon.fr }.
Software engineer, info-dev team.
CREATIS, CNRS UMR 5220, INSERM U1044.
Université Lyon 1, INSA Lyon. France.

\textbf{M. Hern\'andez}
\href{mailto:marc-her@uniandes.edu.co}{marc-her@uniandes.edu.co}
Associate Professor.
IMAGINE
Universidad de los Andes, Bogot\'a D.C. Colombia.

\textbf{M. Orkisz}
\href{mailto:maciej.orkisz@creatis.insa-lyon.fr }{maciej.orkisz@creatis.insa-lyon.fr }.
Associate Professor.
CREATIS, CNRS UMR 5220, INSERM U1044.
Université Lyon 1, INSA Lyon. France.

\section{Introduction}
This software is a computer program whose purpose is to evaluate the performance of different supervised based learning algorithms in the context of image processing (and more particularly on CTA images).
The software has been designed for two main purposes:

Firstly, creaCoroML allows you to use six different supervised learning methods. These methods have been chosen in order to works with a wide range of level-sets. You can select for instance classical methods such as AdaboostM1, or more recent approaches such as the one developed by Seiffert et al.
 
Finally, the software allows you to compare the performance of the three methods on real CTA coronary vessels (MICCAI2012 datasets). The performance can be evaluated from measurements (e.g. using the confusion matrix, sensitivity or specificity) between a reference and the results of the prediction.

\section{Software Presentation}

The software is divided into 4 parts. We can identify:
\begin{itemize}
\item Feature extraction and pre-processing tools.
\item Training and model generation code.
\item Model comparison scripts.
\item Interactive prediction and export of results code.
\end{itemize}

\subsection{System Requirements}

\begin{itemize}
\item At least 2Gb of ram (due to the size of CTA images).
\item Windows or Linux operating systems.
\item CreaTools Framework.
\item At least Matlab R2012b with the following toolboxes: Optimization toolbox, statistics toolbox, curve fitting toolbox, image processing toolbox, ReadData3D\_version1k\/ mha (package to read mhd files).
\item Example files\footnote{(Optional\*, check the wiki site for further information: \href{http://vip.creatis.insa-lyon.fr:9002/projects/creacurvedplanar/files}{http://vip.creatis.insa-lyon.fr:9002/projects/creacurvedplanar/files}.}.
\end{itemize}

\subsection{CreaTools Package Requirements}

\begin{itemize}
\item creaCoro
\item creaVascularTree
\end{itemize}

\subsection{Matlab files}

\begin{itemize}
\item preprocessing.m
\item randomForest\_Demo.m
\item RUSBoost\_Demo.m
\item AdaBoostM1\_Demo.m
\item modelBenchmarking.m
\item InteractivePrediction\_Demo.m
\item src\/OstDistance.m
\item ReadData3D\_version1k\/mha (package to read mhd files)
\end{itemize}

\section{Feature Extraction}

\textbf{File(s)}
\begin{itemize}
\item preprocessing.m
\end{itemize}

\textbf{Inputs}
\begin{itemize}
\item Training Folder (reference file, \* .mhd, \* .raw).
\item Testing Folder (reference file, \* .mhd, \* .raw).
\end{itemize}

\textbf{Outputs}
\begin{itemize}
\item Mat file containing train (trainData) and testing (testData) Data and its labels (yTrain, yTest). By default the matrix is called featureMatrix and its saved at tmp\_data folder.
\end{itemize}

\textbf{Description}

In this step, all CPR (previously generated) vessels are pre-processed by the feature. It surfs through the centerline control points; control points and limits are defined automatically taking care of clinically relevant vessel structures. See chapter \ref{dect:dect} for detailed information.

Also, you can generate by yourself the CPR, using the bbg file (CPR\_MICCAI12) in the bbEditor environment (See Fig. \ref{fig:cpr}). You will need the initial image (.mhd) and the 3D coordinates of the centerline (.txt).

\begin{figure}[ht]
	\centering
		\includegraphics[width=0.7\textwidth]{./Figures/apb_cpr.png}
		\includegraphics[width=0.7\textwidth]{./Figures/apb_cpr2.png}
	\caption[CPR Extraction]{(\textit{Top}) CPR Extraction. (\textit{Bottom}) bbEditor Environment (source: CreaTools screenshot).}
	\label{fig:cpr}
\end{figure}

Parameters like minimum and maximum radius; height of the cylinder and the equi-distal angle of the pattern can be customized, changing the size of the training feature array per point. 

\section{Model Generation}

\textbf{File(s)}
\begin{itemize}
\item randomForest\_Demo.m
\item RUSBoost\_Demo.m
\item AdaBoostM1\_Demo.m
\end{itemize}

\textbf{Inputs}
\begin{itemize}
\item tmp\_data folder (feature matrix file \*.mat).
\end{itemize}

\textbf{Outputs}
\begin{itemize}
\item models\/<modelUsed>.mat
\end{itemize}

\textbf{Description}

Three separated learning methods are provided to build the model.
The classification model uses 250 hundred trees or weak decision learners and a feature space of 289 dimensions equivalent to a cylinder of 5 slices. The training computation time over the entire training dataset is about \~ 13minutes with this configuration.

\section{Model Comparison}

\textbf{File(s)}
\begin{itemize}
\item modelBenchmarking.m
\end{itemize}

\textbf{Inputs}
\begin{itemize}
\item tmp\_results folder (feature matrix file \* .mat).
\item Models folder (RUSBoost,randomForest,AdaBoostM1 .mat).
\end{itemize}

\textbf{Description}

Comparison code is available in this package; you can check methods performance addressing the following measures over the same dataset (See section \ref{eval:eval}).:

\begin{itemize}
\item Classification error
\item ROC curve
\item Sensitivity
\item Specificity
\item Accuracy
\item Positive per value
\item Negative per value
\end{itemize}

Automatic plots are done superposing the information of the three methods (See Fig. \ref{fig:errcurve} and Fig. \ref{fig:roccurve}).

\begin{figure}[ht]
	\centering
		\includegraphics[width=0.5\textwidth]{./Figures/289Final.png}
	\caption[Classification Curve]{Classification error curve.}
	\label{fig:errcurve}
\end{figure}

\begin{figure}[ht]
	\centering
		\includegraphics[width=0.5\textwidth]{./Figures/rocFinal.png}
	\caption[ROC Curve]{ROC Curve.}
	\label{fig:roccurve}
\end{figure}

\section{Interactive Prediction}

\textbf{File(s)}
\begin{itemize}
\item InteractivePrediction\_Demo.m
\end{itemize}

\textbf{Inputs}
\begin{itemize}
\item Testing vessels folder from frontAlgorithms package.
\item Models folder (RUSBoost, randomForest, AdaBoostM1 .mat).
\end{itemize}

\textbf{Outputs}
\begin{itemize}
\item Text file with 3D points and label.
\item Text file with 3D representative lesion points.
\item Reference file (optional).
\end{itemize}

\textbf{Description}

The interactive prediction script brings you the tools to interact with creaCoro and frontAlgorithms packages. It loads a CPR image previously extracted in frontAlgorithms package, then, it predicts calcified lesions. Exporting results are available to plot reference and prediction in creaCoro package. A basic GUI is provide to select datasets and vessels to predict (See Fig. \ref{fig:int1}). In this case, only one vessel at a time is possible to predict.

\begin{figure}[ht]
	\centering
		\includegraphics[width=0.5\textwidth]{./Figures/apb_int1.png}
		\includegraphics[width=0.3\textwidth]{./Figures/apb_int2.png}
	\caption[Dataset Selection GUI]{Dataset and vessel selection GUI.}
	\label{fig:int1}
\end{figure}

You can check results using the bbg file (creaCoroComplexBox\_rusboost) and bbEditor. You will need the image (.mhd) and the prediction output files (.txt). Reference data superposition is possible, since the application receive both files (prediction and reference) in order to visually verify the detection (See Fig. \ref{fig:apb_gui}).

\begin{figure}[ht]
	\centering
		\includegraphics[width=0.9\textwidth]{./Figures/apb_gui.png}
	\caption[Lesion Visualization GUI]{Full creaCoro GUI superposition using optional reference file (source: CreaTools screenshot).}
	\label{fig:apb_gui}
\end{figure} % Appendix Title

%\input{./Appendices/AppendixC} % Appendix Title
\end{appendices}

\addtocontents{toc}{\vspace{2em}}  % Add a gap in the Contents, for aesthetics
\backmatter

%% ----------------------------------------------------------------
\label{References}
\renewcommand\bibname{References}
\lhead{\emph{References}}  % Change the left side page header to "Bibliography"
\bibliographystyle{unsrtnat}  % Use the "unsrtnat" BibTeX style for formatting the Bibliography
%\bibliographystyle{plain}  % Use the "unsrtnat" BibTeX style for formatting the Bibliography
\bibliography{References}  % The references (bibliography) information are stored in the file named "Bibliography.bib"
%% ----------------------------------------------------------------


%% ----------------------------------------------------------------
% The " Bio Page"
\pagestyle{fancy}
\addtocontents{toc}{\vspace{1em}}  % Add a gap in the Contents, for aesthetics
\lhead{\emph{About the author}} 
\chapter{About the author}

\begin{figure}[h]
	\centering
		\includegraphics[width=0.3\textwidth]{./Figures/Perfil.jpg}
	\label{fig:perfil}
\end{figure}

Esteban Correa was born in Pereira, Colombia, in 1987. He received his B.E. degree in computer science from Universidad Tecnol\'ogica de Pereira (Pereira, Colombia), in 2009. During his B.E. Esteban worked in digital design and image processing at Sirius HPC Lab. In 2010, he was awarded a young researcher scholarship from The Administrative Department of Science, Technology and Innovation of Colombia - COLCIENCIAS. In 2011, he started his M.Sc at Universidad de los Andes (Bogot\'a D.C., Colombia). He joined to the Visual Computing Research Group - IMAGINE to work in HCI and image processing in medical imaging.
Currently, He is doing his internship at the CREATIS-UMR at INSA Lyon, France. Also, He is pursuing a joint degree between Universit\'{e} Claude Bernard (Lyon, France) and Universidad de los Andes(Bogot\'a D.C., Colombia) focused on medical imaging.
His research interests lie in the overlap of image processing, machine learning and computer graphics.

%% ----------------------------------------------------------------
\end{document}  % The End
%% ----------------------------------------------------------------