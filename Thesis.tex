%% ----------------------------------------------------------------
%% Thesis.tex -- MAIN FILE (the one that you compile with LaTeX)
%% ---------------------------------------------------------------- 

% Set up the document
\documentclass[a4paper, 11pt, oneside]{Thesis}  % Use the "Thesis" style, based on the ECS Thesis style by Steve Gunn
\graphicspath{{Figures/}}  % Location of the graphics files (set up for graphics to be in PDF format)

% Include any extra LaTeX packages required
%\usepackage[square, numbers, comma, sort&compress]{natbib}  % Use the "Natbib" style for the references in the Bibliography
\usepackage[comma, sort&compress]{natbib}  % Use the "Natbib" style for the references in the Bibliography
\usepackage{verbatim}  % Needed for the "comment" environment to make LaTeX comments
\usepackage{vector}  % Allows "\bvec{}" and "\buvec{}" for "blackboard" style bold vectors in maths
\usepackage{algpseudocode}
\usepackage{algorithm}
\hypersetup{urlcolor=blue, colorlinks=true}  % Colours hyperlinks in blue, but this can be distracting if there are many links.
%\usepackage{hyperref}
%% ----------------------------------------------------------------
\begin{document}
\frontmatter	  % Begin Roman style (i, ii, iii, iv...) page numbering

% Set up the Title Page
\title  {NO NAME YET}
\authors  {\texorpdfstring
            {\href{mailto:Esteban.Correa@creatis.insa-lyon.fr}{Esteban Mauricio Correa Agudelo}}
            {Esteban Mauricio Correa Agudelo}
            }
\addresses  {\groupname\\\deptname\\\univname}  % Do not change this here, instead these must be set in the "Thesis.cls" file, please look through it instead
\date       {\today}
\subject    {}
\keywords   {}

\maketitle
%% ----------------------------------------------------------------

\setstretch{1.3}  % It is better to have smaller font and larger line spacing than the other way round

% Define the page headers using the FancyHdr package and set up for one-sided printing
\fancyhead{}  % Clears all page headers and footers
\rhead{\thepage}  % Sets the right side header to show the page number
\lhead{}  % Clears the left side page header

\pagestyle{fancy}  % Finally, use the "fancy" page style to implement the FancyHdr headers

%% ----------------------------------------------------------------
% Declaration Page required for the Thesis, your institution may give you a different text to place here
%\Declaration{
%
%\addtocontents{toc}{\vspace{1em}}  % Add a gap in the Contents, for aesthetics
%
%I, AUTHOR NAME, declare that this thesis titled, `THESIS TITLE' and the work presented in it are my own. I confirm that:
%
%\begin{itemize} 
%\item[\tiny{$\blacksquare$}] This work was done wholly or mainly while in candidature for a research degree at this University.
% 
%\item[\tiny{$\blacksquare$}] Where any part of this thesis has previously been submitted for a degree or any other qualification at this University or any other institution, this has been clearly stated.
% 
%\item[\tiny{$\blacksquare$}] Where I have consulted the published work of others, this is always clearly attributed.
% 
%\item[\tiny{$\blacksquare$}] Where I have quoted from the work of others, the source is always given. With the exception of such quotations, this thesis is entirely my own work.
% 
%\item[\tiny{$\blacksquare$}] I have acknowledged all main sources of help.
% 
%\item[\tiny{$\blacksquare$}] Where the thesis is based on work done by myself jointly with others, I have made clear exactly what was done by others and what I have contributed myself.
%\\
%\end{itemize}
% 
% 
%Signed:\\
%\rule[1em]{25em}{0.5pt}  % This prints a line for the signature
% 
%Date:\\
%\rule[1em]{25em}{0.5pt}  % This prints a line to write the date
%}
%\clearpage  % Declaration ended, now start a new page

%% ----------------------------------------------------------------
% The " Quote Page"
\pagestyle{empty}  % No headers or footers for the following pages

\null\vfill
% Now comes the " Quote", written in italics
\textit{``La perseverancia y el esfuerzo \\ siempre ser\'{a}n recompensados con el \'{e}xito ...''}
%\\
%\\
%--------------------
%
%\begin{tabbing}
%\textit{On}\= \textit{ce again}, \\ \> \textit{to my family, }\\ \> \textit{to my teachers,} \\ \> \textit{to my friends...}
%\end{tabbing}

\begin{flushright}
%Extraer algo ...
\end{flushright}


\vfill\vfill\vfill\vfill\vfill\vfill\null
\clearpage  % Funny Quote page ended, start a new page
%% ----------------------------------------------------------------

% The Abstract Page
\addtotoc{Abstract}  % Add the "Abstract" page entry to the Contents
\abstract{
\addtocontents{toc}{\vspace{1em}}  % Add a gap in the Contents, for aesthetics

%The Thesis Abstract is written here (and usually kept to just this page). The page is kept centered vertically so can expand into the blank space above the title too\ldots

%Furthermore, coronary arteries play an important role in cardiovascular system and its diseases are mainly caused by the development and accumulation of plaques in artery walls. 

Cardiovascular diseases remain to be the first cause of mortality in the world. In order to detect and quantify lesions in coronary arteries, a conservative approach is commonly followed, beginning with the localization of coronary starting points (ostia) and the extraction of the vessel centerline. This data is usually required to perform the lumen segmentation and helps lesion detection and stenosis quantification tasks. Due to its social, medical, and academic interest, this thesis presents some contributions to the study of coronary arteries in computed tomography angiography images, specially in the centerline extraction and lumen segmentation steps.

First, we describe an axis extraction method developed in our team and we propose an algorithm to correct the centerlines. New axes seemed to be better in healthy sections, but still having problems close to calcified plaques and bifurcations. Then, we present six lumen segmentation methods falling into classical and classification approaches. Best qualitative results were obtained by methods based on region-growing and 2D active contours. In spite of a low Dice score using k-means clustering (66.48\%) compared to the result with a basic thresholding (65.20\%), the study shows some difficulties to validate with the reference data. This is caused by the presence of some regions that make part of the lumen, but were not are labeled as such and are therefore considered as false positives, e.g. a part of the aorta near the ostia. Additionally, all the techniques present important drawbacks due to the difficulty in expressing an explicit geometrical model and a generic intensity profile of coronary arteries, specially in bifurcations and lesions. Trying to avoid this explicit modeling, we used Support Vector Machines, considered as a powerful generic classification method, but we obtained poor segmentation results with Dice scores under 50\% and a considerable increase in computational costs during training phases. This is probably caused by a feature selection unsuited to describe the lumen. Finally, additional future efforts must be made in order to define a complete evaluation framework that  permits the method's comparison and the complete analysis in coronary arteries. 

}

\clearpage  % Abstract ended, start a new page
%% ----------------------------------------------------------------

% The Abstract Page
\addtotoc{R\'{e}sum\'{e}}  % Add the "Abstract" page entry to the Contents
\resume{
\addtocontents{toc}{\vspace{1em}}  % Add a gap in the Contents, for aesthetics

%The Thesis Abstract is written here (and usually kept to just this page). The page is kept centered vertically so can expand into the blank space above the title too\ldots

Les maladies cardiovasculaires demeurent la premi\`{e}re cause de mortalit\'{e} dans le monde. Pour d\'{e}tecter et quantifier les l\'{e}sions dans les art\`{e}res coronaires, l'approche classique est de commencer par localiser de points de d\'{e}part des art\`{e}res (ostia) et d'extraire la ligne centrale (axe) du vaisseau. Ces donn\'{e}es sont n\'{e}cessaires pour ex\'{e}cuter la segmentation du lumen vasculaire et elles aident \`{a} la r\'{e}alisation des t\^{a}ches suivantes comme la d\'{e}tection de l\'{e}sions et la quantification de st\'{e}noses. En raison de l'int\'{e}r\^{e}t social, m\'{e}dical et acad\'{e}mique de ce sujet, ce m\'{e}moire pr\'{e}sente quelques contributions \`{a} l'\'{e}tude d'art\`{e}res coronaires dans des images d'angiographie par tomodensitom\'{e}trique, particuli\`{e}rement dans l'extraction des axes et la segmentation du lumen.

Dans un premier temps, nous d\'{e}crivons une m\'{e}thode d'extraction des axes d\'{e}velopp\'{e}e dans notre groupe et nous proposons un algorithme pour le recentrage des axes dans les plans orthogonaux \`{a} leurs directions principales. Apr\`{e}s ce traitement, les nouveaux axes ont sembl\'{e} \^{e}tre mieux centr\'{e}s notamment dans le cas des sections saines, avec n\'{e}anmoins des probl\`{e}mes lorsque les axes pr\'{e}sentent des plaques calcifi\'{e}es ou des bifurcations. Suite \`{a} cela, nous pr\'{e}sentons six m\'{e}thodes de segmentation du lumen cat\'{e}goris\'{e}es dans deux groupes principaux: techniques classiques et techniques de classification. Les meilleurs r\'{e}sultats qualitatifs ont \'{e}t\'{e} obtenus par des m\'{e}thodes bas\'{e}es sur des contours actifs et de la croissance de r\'{e}gion. L'\'{e}tude montre quelques difficult\'{e}s pour faire une validation ad\'{e}quate avec les donn\'{e}es de r\'{e}f\'{e}rence, ce qui est une des raisons majeures qui fait que l'on obtient un score de Dice tr\`{e}s bas en utilisant un algorithme des k-moyennes (66.48 \%) score comparable aux r\'{e}sultats obtenus par un seuillage classique (65.20 \%), Ceci est caus\'{e} par la pr\'{e}sence des r\'{e}gions faisant partie du lumen vasculaire e.g. une partie de l'aorte pr\`{e}s de l'ostium, mais n'\'{e}tant pas annot\'{e}es comme telles. Ces r\'{e}gions sont donc reconnues comme des faux positifs. 

\clearpage

Toutes les techniques de segmentation test\'{e}es pr\'{e}sentent des inconv\'{e}nients importants en raison de la difficult\'{e} \`{a} exprimer un mod\`{e}le g\'{e}om\'{e}trique explicite et un profil d'intensit\'{e} g\'{e}n\'{e}rique des art\`{e}res coronaires, particuli\`{e}rement dans le cas de bifurcations et de l\'{e}sions. Afin d'\'{e}viter cette mod\'{e}lisation explicite, nous avons utilis\'{e} la m\'{e}thode de s\'{e}parateurs \`{a} vaste marge, consid\'{e}r\'{e}e comme une technique de classification puissante, mais nous avons obtenu des r\'{e}sultats faibles de segmentation avec un score Dice en dessous de 50\% et une augmentation consid\'{e}rable de co{\^u}ts informatiques pendant les phases d'entrainement. Ceci est probablement d{\^u} \`{a} une s\'{e}lection de descripteurs inadapt\'{e}e pour d\'{e}crire le lumen. Finalement, des efforts suppl\'{e}mentaires doivent \^{e}tre faits pour d\'{e}finir un cadre d'\'{e}valuation complet qui permettrait la comparaison des m\'{e}thodes et l'analyse des art\`{e}res coronaires.

}

\clearpage  % Abstract ended, start a new page
%% ----------------------------------------------------------------

\setstretch{1.3}  % Reset the line-spacing to 1.3 for body text (if it has changed)

% The Acknowledgements page, for thanking everyone
\acknowledgements{
\addtocontents{toc}{\vspace{1em}}  % Add a gap in the Contents, for aesthetics

Firstly, I would like to thank my supervisors Marcela Hern\'{a}ndez and Maciej Orkisz for their support, their patience, for all their advices, and specially for their continuous confidence in my work throughout these years of labor. I would also like to say a big ``Thank you!'' to Maria Alejandra Zuluaga for all her support, she makes part of the persons that I admire the most, including also Marcela and Maciej. 

I would also like to thank professors Jos\'{e} Tiberio Hern\'{a}ndez and Olivier Bernard for having supported the development of my master's studies. To Leonardo Fl\'{o}rez for his recent contribution with the project and my document. To some institutions and their staff that have sponsored this work: Interfacultades, Colciencias, CMIRA Scolarships.

Maintenant, je voudrais dire merci \`{a} toutes les personnes de CREATIS (Xavier, Adeline -- merci pour les corrections!, Claire, info--dev, chercheurs, doctorants, stagiaires, ...) qui m'ont permis de conna{\^i}tre un peu plus la vie du labo. Agradezco a mis amigos colombianos, especialmente Carlos, Vanessa, Alfredo, Eduardo, Halime, Daniela, Nicole, Jenny,... gracias por aguantarme en esos momentos tan dif\'{i}ciles y estresantes. Y a mis amigas mexicanas, por hacerme sentir siempre como si estuviera en mi casa.

Aunque son los \'{u}ltimos en esta lista, son y ser\'{a}n siempre los primeros y a quienes les debo m\'{a}s. Agradezco enormemente a mis padres y a mis hermanas, porque siempre est\'{a}n conmigo dandome toda la energ\'{i}a para lograr mis metas. Tambi\'{e}n a Paola por ser siempre un apoyo enorme e incondicional. Mi familia, esto es especialmente para ustedes!!

}
\clearpage  % End of the Acknowledgements
%% ----------------------------------------------------------------

\pagestyle{fancy}  %The page style headers have been "empty" all this time, now use the "fancy" headers as defined before to bring them back


%% ----------------------------------------------------------------
\lhead{\emph{Contents}}  % Set the left side page header to "Contents"
\tableofcontents  % Write out the Table of Contents

%% ----------------------------------------------------------------
\lhead{\emph{List of Figures}}  % Set the left side page header to "List if Figures"
\listoffigures  % Write out the List of Figures

%% ----------------------------------------------------------------
%\lhead{\emph{List of Tables}}  % Set the left side page header to "List of Tables"
%\listoftables  % Write out the List of Tables

%% ----------------------------------------------------------------

%\setstretch{1.5}  % Set the line spacing to 1.5, this makes the following tables easier to read
%\clearpage  % Start a new page
%\lhead{\emph{Abbreviations}}  % Set the left side page header to "Abbreviations"
%\listofsymbols{ll}  % Include a list of Abbreviations (a table of two columns)
%{
%% \textbf{Acronym} & \textbf{W}hat (it) \textbf{S}tands \textbf{F}or \\
%\textbf{LAH} & \textbf{L}ist \textbf{A}bbreviations \textbf{H}ere \\
%
%}

%% ----------------------------------------------------------------

%\clearpage  % Start a new page
%\lhead{\emph{Physical Constants}}  % Set the left side page header to "Physical Constants"
%\listofconstants{lrcl}  % Include a list of Physical Constants (a four column table)
%{
%% Constant Name & Symbol & = & Constant Value (with units) \\
%Speed of Light & $c$ & $=$ & $2.997\ 924\ 58\times10^{8}\ \mbox{ms}^{-\mbox{s}}$ (exact)\\
%
%}

%% ----------------------------------------------------------------

%\clearpage  %Start a new page
%\lhead{\emph{Symbols}}  % Set the left side page header to "Symbols"
%\listofnomenclature{lll}  % Include a list of Symbols (a three column table)
%{
%% symbol & name & unit \\
%$a$ & distance & m \\
%$P$ & power & W (Js$^{-1}$) \\
%& & \\ % Gap to separate the Roman symbols from the Greek
%$\omega$ & angular frequency & rads$^{-1}$ \\
%}

%% ----------------------------------------------------------------
% End of the pre-able, contents and lists of things
% Begin the Dedication page

%\setstretch{1.3}  % Return the line spacing back to 1.3
%
%\pagestyle{empty}  % Page style needs to be empty for this page
%\dedicatory{For/Dedicated to/To my\ldots}
%
%\addtocontents{toc}{\vspace{2em}}  % Add a gap in the Contents, for aesthetics


%% ----------------------------------------------------------------
\mainmatter	  % Begin normal, numeric (1,2,3...) page numbering
\pagestyle{fancy}  % Return the page headers back to the "fancy" style

% Include the chapters of the thesis, as separate files
% Just uncomment the lines as you write the chapters
\lhead{\emph{Introduction}}  
\chapter{Introduction}
%

This document presents some contributions in two active research topics that helps to understand and quantify cardiovascular diseases (CVD): artery lumen segmentation and plaque detection. This study was mainly focused in coronary arteries due to the social, medical, and academic interest around the world in this subject due to a very high morbidity and mortality related to the coronary heart disease. 

The following subsections introduce the relevant medical context about arteries and their illnesses, the image acquisition techniques used for diagnosis, and the main challenges to extract relevant data. In Chapter 2, a revision of the most recent methods in these topics is presented. Chapter 3 and 4 summarize the methods used and their results. Finally, the conclusions of this work are collected in Chapter 5.

\section{Medical Context}
%

Cardiovascular diseases remain as the leading cause of death in the world. The World Health Organization (WHO) classifies them as the highest health problem with an alarming level (39\%) of deaths (17 millions in 2008) caused by non-communicable diseases among people under age of 70, followed by respiratory and digestive diseases (30\%), cancers (27\%), and others~\citep{WHO2011}. The most common reason of heart attacks and cerebrovascular problems is the blockage of the main arteries caused by the development of the atherosclerotic plaque, which begins by an accumulation of fatty materials in the inner arterial wall layer, reducing the blood flow to myocardium or to the brain.

In order to have a reliable and efficient diagnosis, new imaging techniques have been developed to help radiologists and physicians in early detection and treatment monitoring of CVDs. Nevertheless, these tasks can be tedious, time-consuming and error-prone, even for experienced specialists, requiring the use of robust methods to extract the most important information about diseases and anatomical characteristics. Recent studies have demonstrated that the use of advanced vessel analysis software improves the assessment of vascular diseases and reduces the inter-observer variability~\citep{DiCarli2007, Biermann2012}.

\subsection{Arterial Anatomy and Vascular Diseases}
%

Artery and vein wall has a layer structure composed by the \textit{tunica adventitia} (outermost layer), the \textit{tunica media} and the \textit{tunica intima} (innermost layer). The interior of the vessel enclosed by the \textit{tunica intima} is called the vessel lumen.

Coronary arteries supply the cardiac muscle (myocardium) with oxygenated blood. Two main coronary trees cover the left and right part of this muscle and they start at two locations called \textit{ostia} (Figure \ref{fig:HeartVessels}). The right coronary tree has a principal branch which is usually referred to as the right coronary artery (RCA). On the other side, the left coronary tree begins with a thicker branch called the left coronary artery (LCA), but not too far it divides into the left anterior descending artery (LAD) and the left circumflex artery (LCX). This second tree has an important influence in the correct work of the left ventricle that is responsible for providing all the rest of the body with oxygenated blood. The diameter of coronary arteries varies between 6{\,}mm at ostia locations and 0.5{\,}mm at 20{\,}cm from the starting point.

\begin{figure}[ht]
	\centering
		%\includegraphics[width=3.0in]{./Figures/RadiusEstimation.png}
		\includegraphics[width=0.9\textwidth]{./Figures/HeartVesselsSMALL.png}
	\caption[Coronary arteries anatomy and diseases]{\textit{Left}, heart and main coronary arteries anatomy (source: \href{http://www.jeffersonhospital.org/
Tests-and-Treatments/coronary-artery-bypass-grafting.aspx}{Jefferson University Hospitals}). \textit{Right}, schematic illustration of a vascular disease from \citep{Schaap2010Thesis}.}
	\label{fig:HeartVessels}
\end{figure}

Commonly, CVDs are due to atherosclerosis characterized by the development and accumulation of plaques of different materials (i.e. cholesterol, calcium, fibro-fatty deposits) in artery walls reducing the blood flow. These plaques can be classified in three types: 1) non-calcified plaques, having a lower density compared with the vessel lumen, 2) calcified plaques, presenting high density scores, and 3) mixed plaques that have non-calcified and calcified elements within a single plaque~\citep{Pundziute2007}.

\section{Computer Aided Diagnosis}

The number of imaging studies and the exponential grown of computation power have continued for half a century and is not expected to stop for at least another decade. In addition, the complicated medical image nature explains the interest to develop computer aided systems for medical diagnosis purposes \citep{Ginneken2011}.
Today, computer-aided diagnosis (CAD) systems have been introduced as complementary tools to assist radiologists, focusing their attention to certain image areas where potential lesions could be placed. Interpreting a medical image is extremely challenging, anatomical structures, noise and artifacts, make the image complicated. Therefore CAD systems just serve as decision support tool for doctors in the diagnosis task. The doctor (generally a radiologist) is always responsible for the final interpretation of the entire image.

\subsection{How It Works?}

In general, three main steps can be distinguished in the literature on CAD systems (\citep{Ginneken2001}): 1) pre-processing techniques; 2) segmentation of anatomical structures; and 3) analysis usually aimed to detect a specific kind of abnormality. In pre-processing, calibration, resampling and noise removal are usually performed on scanned images. The anatomic regions division of an image is done in the segmentation step. Finally, different measures (size, intensity, geometry, among others) are done in the analyisis step to provide certain cues to doctors.

In heart disease, multiple CAD systems has been designed aiming to detect and quantify coronary issues, [Goldelberg et al] is a good example of a CAD system used as a first simple triage step in clinical practice. Since the general process pipeline is similar through different medical fields, the literature [CITAR KIRISLI 2012] agrees with two specific workflows for coronary lesion diagnosis (\ref{fig:cad_wf}).

\begin{figure}[htbp]
	\centering
		\includegraphics[width=0.4\textwidth]{./Figures/cad_lesion_workflow.png}
	\caption[Coronary Lesion Detection Workflow]{Coronary Lesion Detection Workflow (source: \href{http://www.bigr.nl/publication/880}{Kirisli's Thesis}).}
	\label{fig:cad_wf}
\end{figure}

\subsection{Evaluation of CAD Systems}

Assesment can be very challenging and is considered a pinnacle of the computer-aided diagnosis and one of the most studied areas in medical image analysis. Quantitative measures (Sensitivity and specificity) are only one way in the CAD systems evaluation spectrum \citep{Ginneken2011}. Sensitivity is determined by the percentage of positive cases well detected and placed by the CAD system. The number of false CAD marks per normal image or case is commonly used for specificity \citep{Castellino2005}. These measurements can be obtained by observing the performance of a CAD system on a set of ‘truth’ cases. The ground truth cases are generally established between multiple radiologist agreedment, adding inter-observer variation to the data. For instance, stand alone sensitivity and specificity measurements are not enough to quantify algorithm performance, even, it becomes more difficult to prove a clinically relevant improvement by a certain method. 
Initiatives like the grand challenges in medical image analysis\footnote{Challenges objectives, rules, and results can be visited in \href{http://www.grand-challenge.org}{http://www.grand-challenge.org}.} address the evaluation challenge in medical imaging. It aims to create standard evaluation frameworks (tools and datasets) to compare proposed algorithms with state of the art methods \citep{Hameeteman2011, Schaap2009, Kirisli2013} , but, it is still an open discussion.

\subsection{Conventional Coronary Angiography}

Conventional coronary angiography (CCA) is the gold standard procedure to diagnose CVDs. In summary, a catheter is inserted through the femoral artery or the hand, then contrast agent is injected into the coronary arteries making them visible to X-ray imaging. The image (angiogram) allows the radiologist detect possible vessel narrowing or plaque formations. However, this method is very invasive requiring high expertise to avoid complications (damages to internal arteries, alterations in pulse, etc.), in addition, it requires a continuous flow of dose radiation and contrast agent to the patient to perform the diagnose.

\begin{figure}[htbp]
	\centering
		\includegraphics[width=0.9\textwidth]{./Figures/CCA_Hortense.png}
	\caption[Conventional Coronary Angiography]{Conventional Coronary Angiography (CCA) (a) Right coronary artery tree. (b) Left coronary artery tree  (source: \href{http://www.bigr.nl/publication/880}{Kirisli's Thesis}).}
	\label{fig:cca_im}
\end{figure}

\subsection{Computed Tomography Angiography }

A computerized tomography (CT) coronary angiogram is an image modality to study the coronary arteries of the heart muscle with blood. Unlike a traditional coronary angiogram, CT angiograms don't use a catheter threaded through the coronary vessels. Instead, a coronary CT angiogram relies on intravenously injected iodine-based contrast fluid to highlight the coronary lumen. Then, a powerful X-ray (source and detector array) machine rotate around the patient 
(See \ref{fig:ct_eq}). Yielding a detailed 3D human heart picture from the X-ray attenuation measured. Although CCA has better image resolution, its low but non-negligible risk of procedure due to its invasive nature [CITAR 189 Schaap] is always a drawback. Therefore, new CT angiograms (64-slice) can be a better choice and it is expected to allow advanced computer-aided diagnosis like lumen morphology, reliable stenosis grading among others in the upcoming years.

\begin{figure}[htbp]
	\centering
		\includegraphics[width=0.5\textwidth]{./Figures/ct_equipment.jpg}
	\caption[Computed tomography equipment]{Computed tomography equipment (source: \href{http://www.radiologyinfo.org/}{RadiologyInfo.org}).}
	\label{fig:ct_eq}
\end{figure}

\section{Main Challenges and Workflow}
%
Despite the possibility to improve the study of arteries thanks to high 3D resolution obtained by CTA, new difficulties appear from the image processing point of view. The following list proposed by Lesage \citep{Lesage2009Thesis} summarizes the main problems in this research area:

\begin{itemize}
	\item Size of data and resolution: CTA images typically reach sizes of $512\times512\times512$ voxels. In the case of coronaries, the radius varies between 1 to 10 voxels.	
	\item Acquisition issues: noise, partial volume effect, and artifacts influence the results.	
	\item Arteries complexity and variability: location, curvatures, sizes, and geometries are different in each patient.
	\item Pathologies and altering objects: presence of plaques, stents and bypasses affect the geometry and appearance.
	\item Surrounding structures: in the case of coronary trees, vein networks and heart chambers can be mistaken for arteries because of their similar appearance or shape. 
\end{itemize}

\begin{figure}[htbp]
	\centering
		\includegraphics[width=0.9\textwidth]{./Figures/Workflow.png}
	\caption[Workflow for coronary arteries analysis.]{Workflow for coronary arteries analysis.}
	\label{fig:Workflow}
\end{figure}

Considering these issues, the analysis of arteries and their pathologies is usually pursued by following a workflow that begins with the localization of some starting seed (generally at \textit{ostia} locations) in order to initialize an axis extraction algorithm to identify the path of the main arteries. Additionally, the lumen segmentation can also be a result of this step or, depending on the approach used, it can give a first approach to the volume of interest to be processed. After these first steps, both lesion detection and stenosis quantification can be achieved in order to present specific and objective data about diseases to physicians (Figure \ref{fig:Workflow}).

\section{Institutional Context}
%
The project is part of a Franco-Colombian scientific collaboration developed by the following research teams: The Medical Imaging Research Center, CREATIS-UMR, Universit\'{e} Claude Bernard Lyon 1, Lyon, France; The Visual Computing Research Group, Imagine,Universidad de los Andes, Bogot\'{a}, Colombia and the Bio-informatics and Computer Graphics Group, Takina, Pontificia Universidad Javeriana, Bogot\'{a}, Colombia. In recent years, these institutions have permanently been working on carotid and coronary artery modeling in CTA images.

Financial support for the projects of this collaboration has been provided by the ECOS Nord Commitee grant (C11S01); the Administrative Department of Science, Technology and Innovation of Colombia - COLCIENCIAS and Uniandes Interfacultades (06-2010). In Addition, this project is under the joint master degree program between the Electric Engineering Department of the University Claude Bernard Lyon 1 (UCBL) and the Systems Engineering Department of the Universidad de los Andes (Uniandes). 

\section{Internship Context}
The work of this internship at CREATIS-UMR is the continuation of my first master year at uniandes. The topic of this tesis was carried out in the context of the ``Stenoses Detection, Quantification and Lumen Segmentation'' challenge. It was held during the 15th International Conference on Medical Image Computing and Computer Assisted Intervention (MICCAI), Nice, France, 1-6 October 2012.
 % Introduction

%\input{./Chapters/02_MedicalContext} % Background Theory 
%\lhead{\emph{Related Work}}  
%\chapter{Related Work}\label{rel:rel}

\section{Centerline Extraction}

In recent years, the number (CAD) systems related to coronary diagnosis has increased \citep{Goldenberg2012}. Most, of the work that is done by these systems, involves a coronary tree extraction. Many axis extraction methods have been evaluated in \citep{Schaap2009} giving high accuracy compared to expert annotations. Here we give an overview of the methods evaluated according to similar techniques, we refer readers to the table \ref{tb:rel_cent} based on Schaap's work for an extensive evaluation and review on extraction methods according to the prior information and interaction levels (more than one point per vessel, one point per vessel, fully automatic).
\citep{Zhang2008, Tek2008, Metz2009, Krissian2008, Szymczak2008, Dikici2008, Friman2008} are vesselness-methods based or make use of a shortest path approach. \citep{Kitslaar2008, Castro2008} are morphology-based approaches. Inertia analysis method was developed by \citep{HernandezHoyos2008}. \citep{Wang2008} uses skeletonization and fuzzy conectedness tree algorithm. Finally, shape and appearance fitting models are used by \citep{Zambal2008}.
Minimum cost path methods that enhance specific structures of interest give better results \citep{Schaap2009}. The first group (shortest path-based methods) finds the coronary tree using one or more points and outperform the other ones.

\begin{table*}
\scriptsize
\caption{Overview of the previously published centerline extraction challenge \citep{Schaap2009}}
\centering
\begin{tabular}{|p{4cm}|p{10cm}|}
\hline
\multicolumn{1}{|p{4cm}|}{} &\multicolumn{1}{p{10cm}|}{\textbf{Description}}\\
\hline
\begin{flushleft}
Minimal user-interaction (More than one point per vessel)
\end{flushleft} & 
\begin{itemize}
	\item Local hessian-based vesselness (Zhang et al.). 
	\item Intensity threshold $+$ inertia analysis (Hern\'andez et al.).
	\item Multiple hypothesis tracking $+$ Fast marching (Friman et al.).
	\item Center of intensity plateaus in 2D slices (Szymczak et al.).
	\item Hessian based vesselness $+$ Minimum cost path (Metz et al..).
\end{itemize}\\
\hline
\begin{flushleft}
Semi-automatic (One point per vessel)
\end{flushleft} &
\begin{itemize}
	\item Hough-like $+$ minimum cost path (Dikici et al.).
	\item Iterative 3D morphology (Castro et al.).
	\item Feature space and vesselness $+$ Minimum cost path (Krissian et al.)
\end{itemize}\\
\hline
\begin{flushleft}
Full automatic
\end{flushleft} &
\begin{itemize}
	\item Multi-scale medialness based (Tek et al.).
	\item Segmentation and connected components (Kitslaar et al.).
	\item Shape and appearance fitting models (Zambal et al.).
	\item Fuzzy conectedness tree algorithm (Wang et al.).
\end{itemize}\\
\hline
\end{tabular}
\vspace{-0.3cm}
\label{tb:rel_cent}
\normalsize
\end{table*}

\section{Coronary Lesion Detection}

Here, we give an overview of learning-based methods for plaque detection in CTA images. We refer readers to  \citep{Kirisli2013} for an extensive evaluation and review on lesion detection.

Coronary lesions have no specific shape, size or position along the axis, they can show-up anywhere there is a vessel. An appropriate pattern should therefore be defined to capture the entire information of the lesion \citep{Zuluaga2011c} defined a 2D cross sectional pattern to estimate the lesion density. \citep{Tessmann2009} proposed a 3D multi-scale cylindrical pattern, where they extracted sample positions on a cylinder for feature extraction. Similar pattern was used by \citep{Mittal2010} but using steerable features. Finally, \citep{Cetin2012} proposed region based cylindrical with a vessel intensity approach.

In addition, small amount of illness labeled data versus no illness data add an extra difficulty in terms of feature pre-processing in the learning method.\citep{Chawla2002} proposed synthetic over-sampling in the feature space to balance datasets before a boosting classification. They propose an over-sampling approach in which the minority class is over-sampled by creating “synthetic” examples rather than by over-sampling with replacement. \citep{Seiffert2010} undersamples the majority class(es) to do the same but in faster and simple way.
In coronary lesions, \citep{Mittal2010} apply a synthetic interpolation of lesion slices in conjunction with a 3D sampling pattern for  calcified and non-calcified plaques. \citep{Zuluaga2011Thesis} adress this problem from the semi-supervised learning perspective, formuling as a density level detection problem with no labeled data available but with a 2D intensity metric. In the same direction \citep{Zuluaga2011a} has took into consideration only positive and unlabeled slices as a second semi-supervised methodology.
 % Experimental Setup

%\lhead{\emph{Materials and Methods}}  
%\input{./Chapters/03_Method} % Experiment 1

%\lhead{\emph{Results}}  
%\input{./Chapters/04_Results} % Experiment 2

%\lhead{\emph{Conclusions}}  
%\input{./Chapters/05_Conclusions} % Experiment 2

%\input{./Chapters/Chapter6} % Results and Discussion

%\input{./Chapters/Chapter7} % Conclusion

%% ----------------------------------------------------------------
% Now begin the Appendices, including them as separate files

%\addtocontents{toc}{\vspace{2em}} % Add a gap in the Contents, for aesthetics

%\appendix % Cue to tell LaTeX that the following 'chapters' are Appendices

%% Appendix A

\chapter{frontAlgorithms: A Semi-automatic Coronary Centerline Extraction Framework} \label{AppendixA}

\begin{figure}[ht]
	\centering
		\includegraphics[width=1.3\textwidth]{./Figures/extrac6.png}
	\label{fig:apa_cent1}
\end{figure}
\clearpage

\section{Copyright}
Esteban Correa, Universidad de los Andes or CREATIS-UMR do not offer any support for this product whatsoever. The program is offered free of charge. The executable program is copyrighted freeware by CREATIS-UMR.

\section{Contributors}

\textbf{Esteban Correa}
\href{mailto:esteban.correa@creatis.insa-lyon.fr }{esteban.correa@creatis.insa-lyon.fr }
Master Student for centerline extraction methods and development.
Universidad de los Andes, Bogot\'a D.C. Colombia.
Université Lyon 1, France.

\textbf{Leonardo Fl\'orez}
\href{mailto:florez-l@javeriana.edu.co }{florez-l@javeriana.edu.co }
Associate Professor and main software developer and centerline extraction methods.
TAKINA
Pontificia Universidad Javeriana, Bogot\'a D.C. Colombia.

\textbf{M. Orkisz}
\href{mailto:maciej.orkisz@creatis.insa-lyon.fr }{maciej.orkisz@creatis.insa-lyon.fr }
Associate Professor
CREATIS, CNRS UMR 5220, INSERM U1044.
Université Lyon 1, INSA Lyon. France.

\section{Introduction}
This software is a computer program whose purpose is to evaluate the performance of centerline extraction methods in the context of image processing (and more particularly on CTA images).
The software has been designed for two main purposes:

This method finds a coronary vessel centerline interactively. Two seed points are manually placed by the radiologist and the path is automatically extracted using a minimum cost path approach (Dijkstra’s algorithm). The cost to travel through a voxel is based on a generalized logistic function applied to the non-linear flux vesselness response \citep{Lesage2009a}  of the CTA image. Selective vessel radiuses can be provided to get rid off erroneously connected structures such coronary veins. Results shown robust and accurate centerline extraction in multi-vendor datasets.

Finally, the software allows you to compare the performance of vesselness in a shortest path approach. By default, two vesselness methods are provided \citep{Tek2008,Lesage2009a} with their respective cost functions. A graphical interface enables the algorithm progression.

The software is divided into 2 parts. We can identify:
\begin{itemize}
\item Seeds placement GUI.
\item Centerline Extraction Algorithm.
\end{itemize}

\subsection{System Requirements}

\begin{itemize}
\item At least 2Gb of ram (due to the size of CTA images).
\item Windows or Linux operating systems.
\item CreaTools Framework.
\end{itemize}

\subsection{CreaTools Package Requirements}

\begin{itemize}
\item itkRGC\/frontAlgorithms
\end{itemize}

\section{Point placement}

\textbf{File(s)}
\begin{itemize}
\item fa\_example\_extract\_path\_with\_dijkstra\_06.bbg
\end{itemize}

\textbf{Inputs}
\begin{itemize}
\item Image file (\* .mhd, \* .raw).
\end{itemize}

\textbf{Description}

\begin{figure}[ht]
	\centering
		\includegraphics[width=0.7\textwidth]{./Figures/apa_bbe.png}
	\caption[Radiologist's GUI]{(\textit{Top}) Start point (at ostium). (\textit{Bottom}) end point (distal location) (source: CreaTools screenshot).}
	\label{fig:apa_bbe}
\end{figure}

In order to integrate results from frontAlgorithms extraction with creaCoro aplication. We have build a new bbtk diagram to simulate in some way, how the radiologist works on the images (See Fig. \ref{fig:apa_bbe}). Basically, the user (radiologist) select start and end points of the coronary artery in CTA image (See Fig. \ref{fig:apa_sp1}). 

\begin{figure}[ht]
	\centering
		\includegraphics[width=0.8\textwidth]{./Figures/ShowNPoints1.png}
		\includegraphics[width=0.8\textwidth]{./Figures/ShowNPoints1.png}
	\caption[Radiologist's GUI]{(\textit{Top}) Start point (at ostium). (\textit{Bottom}) end point (distal location) (source: CreaTools screenshot).}
	\label{fig:apa_sp1}
\end{figure}

\section{Centerline Extraction}

\textbf{File(s)}
\begin{itemize}
\item fa\_example\_extract\_path\_with\_dijkstra\_06.bbg
\end{itemize}

\textbf{Inputs}
\begin{itemize}
\item Image file (\* .mhd, \* .raw).
\item Vesselness (offline computed) image file (\* .mhd, \* .raw) .
\item Seed point files (\* .txt)
\end{itemize}

\textbf{Outputs}
\begin{itemize}
\item vtk file containing the centerline parametric path.
\end{itemize}

\textbf{Description}

Currently, you can compute two types of vesselness images (Gulsun, MFlux), command line applications(package\/appli\/examples) are provided for this purpose. Once you have your vesselness image, you can set parameters of the cost function (See section \ref{cent:cost} in order to improve the extraction. Three pre-programmed functions are available. Finally you can launch the extraction method and see the progression interatively (See Fig. \ref{fig:apa_cent1}).

\begin{figure}[ht]
	\centering
		\includegraphics[width=0.5\textwidth]{./Figures/extrac1.png}
		\includegraphics[width=0.5\textwidth]{./Figures/extrac2.png}
		\includegraphics[width=0.5\textwidth]{./Figures/extrac3.png}
		\includegraphics[width=0.5\textwidth]{./Figures/extrac4.png}
	\caption[Centerline Extraction GUI 1]{Extraction progression (source: CreaTools screenshot).}
	\label{fig:apa_cent1}
\end{figure}

\begin{figure}[ht]
	\centering
		\includegraphics[width=0.6\textwidth]{./Figures/extrac6.png}
		\includegraphics[width=0.6\textwidth]{./Figures/extrac7.png}
		\includegraphics[width=0.6\textwidth]{./Figures/extrac8.png}
		\includegraphics[width=0.6\textwidth]{./Figures/extrac9.png}
	\caption[Centerline Extraction GUI 2]{Centerline extracted (source: CreaTools screenshot).}
	\label{fig:apa_cent1}
\end{figure}

	% Appendix Title

%% Appendix A

\chapter{CreaCoro: A Coronary Lesion Visualization Framework}\label{AppendixB}

\begin{figure}[ht]
	\centering
		\includegraphics[width=1.0\textwidth]{./Figures/heart3d.png}
	\label{fig:longi_feat}
\end{figure}
\clearpage
\section{Copyright}
Esteban Correa, Universidad de los Andes or CREATIS-UMR do not offer any support for this product whatsoever. The program is offered free of charge. The executable program is copyrighted freeware by CREATIS-UMR.

\section{Contributors}

\textbf{Esteban Correa}
\href{mailto:esteban.correa@creatis.insa-lyon.fr }{esteban.correa@creatis.insa-lyon.fr }.
Master Student for methods and development.
Universidad de los Andes, Bogot\'a D.C. Colombia.
Université Lyon 1, France.

\textbf{Claire Mouton}
\href{mailto:claire.mouton@creatis.insa-lyon.fr}{claire.mouton@creatis.insa-lyon.fr}.
Software engineer, info-dev team.
CREATIS, CNRS UMR 5220, INSERM U1044.
Université Lyon 1, INSA Lyon. France.

\textbf{Eduardo Enrique Davila Serrano}
\href{mailto:eduardo.davila@creatis.insa-lyon.fr }{eduardo.davila@creatis.insa-lyon.fr }.
Software engineer, info-dev team.
CREATIS, CNRS UMR 5220, INSERM U1044.
Université Lyon 1, INSA Lyon. France.

\textbf{M. Hern\'andez}
\href{mailto:marc-her@uniandes.edu.co}{marc-her@uniandes.edu.co}
Associate Professor.
IMAGINE
Universidad de los Andes, Bogot\'a D.C. Colombia.

\textbf{M. Orkisz}
\href{mailto:maciej.orkisz@creatis.insa-lyon.fr }{maciej.orkisz@creatis.insa-lyon.fr }.
Associate Professor.
CREATIS, CNRS UMR 5220, INSERM U1044.
Université Lyon 1, INSA Lyon. France.

\section{Introduction}
This software is a computer program whose purpose is to evaluate the performance of different supervised based learning algorithms in the context of image processing (and more particularly on CTA images).
The software has been designed for two main purposes:

Firstly, creaCoroML allows you to use six different supervised learning methods. These methods have been chosen in order to works with a wide range of level-sets. You can select for instance classical methods such as AdaboostM1, or more recent approaches such as the one developed by Seiffert et al.
 
Finally, the software allows you to compare the performance of the three methods on real CTA coronary vessels (MICCAI2012 datasets). The performance can be evaluated from measurements (e.g. using the confusion matrix, sensitivity or specificity) between a reference and the results of the prediction.

\section{Software Presentation}

The software is divided into 4 parts. We can identify:
\begin{itemize}
\item Feature extraction and pre-processing tools.
\item Training and model generation code.
\item Model comparison scripts.
\item Interactive prediction and export of results code.
\end{itemize}

\subsection{System Requirements}

\begin{itemize}
\item At least 2Gb of ram (due to the size of CTA images).
\item Windows or Linux operating systems.
\item CreaTools Framework.
\item At least Matlab R2012b with the following toolboxes: Optimization toolbox, statistics toolbox, curve fitting toolbox, image processing toolbox, ReadData3D\_version1k\/ mha (package to read mhd files).
\item Example files\footnote{(Optional\*, check the wiki site for further information: \href{http://vip.creatis.insa-lyon.fr:9002/projects/creacurvedplanar/files}{http://vip.creatis.insa-lyon.fr:9002/projects/creacurvedplanar/files}.}.
\end{itemize}

\subsection{CreaTools Package Requirements}

\begin{itemize}
\item creaCoro
\item creaVascularTree
\end{itemize}

\subsection{Matlab files}

\begin{itemize}
\item preprocessing.m
\item randomForest\_Demo.m
\item RUSBoost\_Demo.m
\item AdaBoostM1\_Demo.m
\item modelBenchmarking.m
\item InteractivePrediction\_Demo.m
\item src\/OstDistance.m
\item ReadData3D\_version1k\/mha (package to read mhd files)
\end{itemize}

\section{Feature Extraction}

\textbf{File(s)}
\begin{itemize}
\item preprocessing.m
\end{itemize}

\textbf{Inputs}
\begin{itemize}
\item Training Folder (reference file, \* .mhd, \* .raw).
\item Testing Folder (reference file, \* .mhd, \* .raw).
\end{itemize}

\textbf{Outputs}
\begin{itemize}
\item Mat file containing train (trainData) and testing (testData) Data and its labels (yTrain, yTest). By default the matrix is called featureMatrix and its saved at tmp\_data folder.
\end{itemize}

\textbf{Description}

In this step, all CPR (previously generated) vessels are pre-processed by the feature. It surfs through the centerline control points; control points and limits are defined automatically taking care of clinically relevant vessel structures. See chapter \ref{dect:dect} for detailed information.

Also, you can generate by yourself the CPR, using the bbg file (CPR\_MICCAI12) in the bbEditor environment (See Fig. \ref{fig:cpr}). You will need the initial image (.mhd) and the 3D coordinates of the centerline (.txt).

\begin{figure}[ht]
	\centering
		\includegraphics[width=0.7\textwidth]{./Figures/apb_cpr.png}
		\includegraphics[width=0.7\textwidth]{./Figures/apb_cpr2.png}
	\caption[CPR Extraction]{(\textit{Top}) CPR Extraction. (\textit{Bottom}) bbEditor Environment (source: CreaTools screenshot).}
	\label{fig:cpr}
\end{figure}

Parameters like minimum and maximum radius; height of the cylinder and the equi-distal angle of the pattern can be customized, changing the size of the training feature array per point. 

\section{Model Generation}

\textbf{File(s)}
\begin{itemize}
\item randomForest\_Demo.m
\item RUSBoost\_Demo.m
\item AdaBoostM1\_Demo.m
\end{itemize}

\textbf{Inputs}
\begin{itemize}
\item tmp\_data folder (feature matrix file \*.mat).
\end{itemize}

\textbf{Outputs}
\begin{itemize}
\item models\/<modelUsed>.mat
\end{itemize}

\textbf{Description}

Three separated learning methods are provided to build the model.
The classification model uses 250 hundred trees or weak decision learners and a feature space of 289 dimensions equivalent to a cylinder of 5 slices. The training computation time over the entire training dataset is about \~ 13minutes with this configuration.

\section{Model Comparison}

\textbf{File(s)}
\begin{itemize}
\item modelBenchmarking.m
\end{itemize}

\textbf{Inputs}
\begin{itemize}
\item tmp\_results folder (feature matrix file \* .mat).
\item Models folder (RUSBoost,randomForest,AdaBoostM1 .mat).
\end{itemize}

\textbf{Description}

Comparison code is available in this package; you can check methods performance addressing the following measures over the same dataset (See section \ref{eval:eval}).:

\begin{itemize}
\item Classification error
\item ROC curve
\item Sensitivity
\item Specificity
\item Accuracy
\item Positive per value
\item Negative per value
\end{itemize}

Automatic plots are done superposing the information of the three methods (See Fig. \ref{fig:errcurve} and Fig. \ref{fig:roccurve}).

\begin{figure}[ht]
	\centering
		\includegraphics[width=0.5\textwidth]{./Figures/289Final.png}
	\caption[Classification Curve]{Classification error curve.}
	\label{fig:errcurve}
\end{figure}

\begin{figure}[ht]
	\centering
		\includegraphics[width=0.5\textwidth]{./Figures/rocFinal.png}
	\caption[ROC Curve]{ROC Curve.}
	\label{fig:roccurve}
\end{figure}

\section{Interactive Prediction}

\textbf{File(s)}
\begin{itemize}
\item InteractivePrediction\_Demo.m
\end{itemize}

\textbf{Inputs}
\begin{itemize}
\item Testing vessels folder from frontAlgorithms package.
\item Models folder (RUSBoost, randomForest, AdaBoostM1 .mat).
\end{itemize}

\textbf{Outputs}
\begin{itemize}
\item Text file with 3D points and label.
\item Text file with 3D representative lesion points.
\item Reference file (optional).
\end{itemize}

\textbf{Description}

The interactive prediction script brings you the tools to interact with creaCoro and frontAlgorithms packages. It loads a CPR image previously extracted in frontAlgorithms package, then, it predicts calcified lesions. Exporting results are available to plot reference and prediction in creaCoro package. A basic GUI is provide to select datasets and vessels to predict (See Fig. \ref{fig:int1}). In this case, only one vessel at a time is possible to predict.

\begin{figure}[ht]
	\centering
		\includegraphics[width=0.5\textwidth]{./Figures/apb_int1.png}
		\includegraphics[width=0.3\textwidth]{./Figures/apb_int2.png}
	\caption[Dataset Selection GUI]{Dataset and vessel selection GUI.}
	\label{fig:int1}
\end{figure}

You can check results using the bbg file (creaCoroComplexBox\_rusboost) and bbEditor. You will need the image (.mhd) and the prediction output files (.txt). Reference data superposition is possible, since the application receive both files (prediction and reference) in order to visually verify the detection (See Fig. \ref{fig:apb_gui}).

\begin{figure}[ht]
	\centering
		\includegraphics[width=0.9\textwidth]{./Figures/apb_gui.png}
	\caption[Lesion Visualization GUI]{Full creaCoro GUI superposition using optional reference file (source: CreaTools screenshot).}
	\label{fig:apb_gui}
\end{figure} % Appendix Title

%\input{./Appendices/AppendixC} % Appendix Title

\addtocontents{toc}{\vspace{2em}}  % Add a gap in the Contents, for aesthetics
\backmatter

%% ----------------------------------------------------------------
\label{References}
\renewcommand\bibname{References}
\lhead{\emph{References}}  % Change the left side page header to "Bibliography"
\bibliographystyle{unsrtnat}  % Use the "unsrtnat" BibTeX style for formatting the Bibliography
\bibliography{References}  % The references (bibliography) information are stored in the file named "Bibliography.bib"

\end{document}  % The End
%% ----------------------------------------------------------------